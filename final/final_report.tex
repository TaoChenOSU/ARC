\documentclass{article}
\usepackage[utf8]{inputenc}

\title{Capstone Final Report}
\author{Dan Stoyer, Cierra Shawe, Tao Chen }
\date{June 2017}

\begin{document}

\maketitle

\section{Introduction}
[Who requested it?][Why was it requested?][What was its importance?][Who was your client?][Who are the members of our team?][What were their roles?][What was the role of the client? (i.e. supervision only, participate in developement, etc.)]

\section{Original Requirements Document}

\section{Requirements Changes}

\section{Design Document}

    \subsection{Changes Made}

\section{Tech Review}

    \subsection{Changes Made}

\section{Weekly Blogs}

\subsection{Cierra's Blog Entries}

\subsubsection{Fall Week 4}
\begin{itemize}
    \item {\textbf{Worked On}}
    \begin{itemize}
        
    \end{itemize}

    \item {\textbf{Problems Encountered}}
    \begin{itemize}
        
    \end{itemize}

    \item{\textbf{Plans for next week}}
    \begin{itemize}
        
    \end{itemize}

\end{itemize}

\subsubsection{Fall Week 5}
\begin{itemize}
    \item {\textbf{Worked On}}
    \begin{itemize}
        
    \end{itemize}

    \item {\textbf{Problems Encountered}}
    \begin{itemize}
        
    \end{itemize}

    \item{\textbf{Plans for next week}}
    \begin{itemize}
        
    \end{itemize}

\end{itemize}

\subsubsection{Fall Week 6}
\begin{itemize}
    \item {\textbf{Worked On}}
    \begin{itemize}
        
    \end{itemize}

    \item {\textbf{Problems Encountered}}
    \begin{itemize}
        
    \end{itemize}

    \item{\textbf{Plans for next week}}
    \begin{itemize}
        
    \end{itemize}

\end{itemize}

\subsubsection{Fall Week 7}
\begin{itemize}
    \item {\textbf{Worked On}}
    \begin{itemize}
        
    \end{itemize}

    \item {\textbf{Problems Encountered}}
    \begin{itemize}
        
    \end{itemize}

    \item{\textbf{Plans for next week}}
    \begin{itemize}
        
    \end{itemize}

\end{itemize}

\subsubsection{Fall Week 8}
\begin{itemize}
    \item {\textbf{Worked On}}
    \begin{itemize}
        
    \end{itemize}

    \item {\textbf{Problems Encountered}}
    \begin{itemize}
        
    \end{itemize}

    \item{\textbf{Plans for next week}}
    \begin{itemize}
        
    \end{itemize}

\end{itemize}

\subsubsection{Fall Week 9}
\begin{itemize}
    \item {\textbf{Worked On}}
    \begin{itemize}
        
    \end{itemize}

    \item {\textbf{Problems Encountered}}
    \begin{itemize}
        
    \end{itemize}

    \item{\textbf{Plans for next week}}
    \begin{itemize}
        
    \end{itemize}

\end{itemize}

\subsubsection{Fall Week 10}
\begin{itemize}
    \item {\textbf{Worked On}}
    \begin{itemize}
        
    \end{itemize}

    \item {\textbf{Problems Encountered}}
    \begin{itemize}
        
    \end{itemize}

    \item{\textbf{Plans for next week}}
    \begin{itemize}
        
    \end{itemize}

\end{itemize}

\subsubsection{Fall Week 11}
\begin{itemize}
    \item {\textbf{Worked On}}
    \begin{itemize}
        
    \end{itemize}

    \item {\textbf{Problems Encountered}}
    \begin{itemize}
        
    \end{itemize}

    \item{\textbf{Plans for next week}}
    \begin{itemize}
        
    \end{itemize}

\end{itemize}

\subsubsection{Winter Week 1}
\begin{itemize}
    \item {\textbf{Worked On}}
    \begin{itemize}
        
    \end{itemize}

    \item {\textbf{Problems Encountered}}
    \begin{itemize}
        
    \end{itemize}

    \item{\textbf{Plans for next week}}
    \begin{itemize}
        
    \end{itemize}

\end{itemize}

\subsubsection{Winter Week 2}
\begin{itemize}
    \item {\textbf{Worked On}}
    \begin{itemize}
        
    \end{itemize}

    \item {\textbf{Problems Encountered}}
    \begin{itemize}
        
    \end{itemize}

    \item{\textbf{Plans for next week}}
    \begin{itemize}
        
    \end{itemize}

\end{itemize}

\subsubsection{Winter Week 3}
\begin{itemize}
    \item {\textbf{Worked On}}
    \begin{itemize}
        
    \end{itemize}

    \item {\textbf{Problems Encountered}}
    \begin{itemize}
        
    \end{itemize}

    \item{\textbf{Plans for next week}}
    \begin{itemize}
        
    \end{itemize}

\end{itemize}

\subsubsection{Winter Week 4}
\begin{itemize}
    \item {\textbf{Worked On}}
    \begin{itemize}
        
    \end{itemize}

    \item {\textbf{Problems Encountered}}
    \begin{itemize}
        
    \end{itemize}

    \item{\textbf{Plans for next week}}
    \begin{itemize}
        
    \end{itemize}

\end{itemize}

\subsubsection{Winter Week 5}
\begin{itemize}
    \item {\textbf{Worked On}}
    \begin{itemize}
        
    \end{itemize}

    \item {\textbf{Problems Encountered}}
    \begin{itemize}
        
    \end{itemize}

    \item{\textbf{Plans for next week}}
    \begin{itemize}
        
    \end{itemize}

\end{itemize}

\subsubsection{Winter Week 6}
\begin{itemize}
    \item {\textbf{Worked On}}
    \begin{itemize}
        
    \end{itemize}

    \item {\textbf{Problems Encountered}}
    \begin{itemize}
        
    \end{itemize}

    \item{\textbf{Plans for next week}}
    \begin{itemize}
        
    \end{itemize}

\end{itemize}

\subsubsection{Winter Week 7}
\begin{itemize}
    \item {\textbf{Worked On}}
    \begin{itemize}
        
    \end{itemize}

    \item {\textbf{Problems Encountered}}
    \begin{itemize}
        
    \end{itemize}

    \item{\textbf{Plans for next week}}
    \begin{itemize}
        
    \end{itemize}

\end{itemize}

\subsubsection{Winter Week 8}
\begin{itemize}
    \item {\textbf{Worked On}}
    \begin{itemize}
        
    \end{itemize}

    \item {\textbf{Problems Encountered}}
    \begin{itemize}
        
    \end{itemize}

    \item{\textbf{Plans for next week}}
    \begin{itemize}
        
    \end{itemize}

\end{itemize}

\subsubsection{Winter Week 9}
\begin{itemize}
    \item {\textbf{Worked On}}
    \begin{itemize}
        
    \end{itemize}

    \item {\textbf{Problems Encountered}}
    \begin{itemize}
        
    \end{itemize}

    \item{\textbf{Plans for next week}}
    \begin{itemize}
        
    \end{itemize}

\end{itemize}

\subsubsection{Winter Week 10}
\begin{itemize}
    \item {\textbf{Worked On}}
    \begin{itemize}
        
    \end{itemize}

    \item {\textbf{Problems Encountered}}
    \begin{itemize}
        
    \end{itemize}

    \item{\textbf{Plans for next week}}
    \begin{itemize}
        
    \end{itemize}

\end{itemize}

\subsubsection{Winter Week 11}
\begin{itemize}
    \item {\textbf{Worked On}}
    \begin{itemize}
        
    \end{itemize}

    \item {\textbf{Problems Encountered}}
    \begin{itemize}
        
    \end{itemize}

    \item{\textbf{Plans for next week}}
    \begin{itemize}
        
    \end{itemize}

\end{itemize}

\subsubsection{Spring Week 1}
\begin{itemize}
    \item {\textbf{Worked On}}
    \begin{itemize}
        
    \end{itemize}

    \item {\textbf{Problems Encountered}}
    \begin{itemize}
        
    \end{itemize}

    \item{\textbf{Plans for next week}}
    \begin{itemize}
        
    \end{itemize}

\end{itemize}

\subsubsection{Spring Week 2}
\begin{itemize}
    \item {\textbf{Worked On}}
    \begin{itemize}
        
    \end{itemize}

    \item {\textbf{Problems Encountered}}
    \begin{itemize}
        
    \end{itemize}

    \item{\textbf{Plans for next week}}
    \begin{itemize}
        
    \end{itemize}

\end{itemize}

\subsubsection{Spring Week 3}
\begin{itemize}
    \item {\textbf{Worked On}}
    \begin{itemize}
        
    \end{itemize}

    \item {\textbf{Problems Encountered}}
    \begin{itemize}
        
    \end{itemize}

    \item{\textbf{Plans for next week}}
    \begin{itemize}
        
    \end{itemize}

\end{itemize}

\subsubsection{Spring Week 4}
\begin{itemize}
    \item {\textbf{Worked On}}
    \begin{itemize}
        
    \end{itemize}

    \item {\textbf{Problems Encountered}}
    \begin{itemize}
        
    \end{itemize}

    \item{\textbf{Plans for next week}}
    \begin{itemize}
        
    \end{itemize}

\end{itemize}

\subsubsection{Spring Week 5}
\begin{itemize}
    \item {\textbf{Worked On}}
    \begin{itemize}
        
    \end{itemize}

    \item {\textbf{Problems Encountered}}
    \begin{itemize}
        
    \end{itemize}

    \item{\textbf{Plans for next week}}
    \begin{itemize}
        
    \end{itemize}

\end{itemize}

\subsubsection{Spring Week 6}
\begin{itemize}
    \item {\textbf{Worked On}}
    \begin{itemize}
        
    \end{itemize}

    \item {\textbf{Problems Encountered}}
    \begin{itemize}
        
    \end{itemize}

    \item{\textbf{Plans for next week}}
    \begin{itemize}
        
    \end{itemize}

\end{itemize}

\subsubsection{Spring Week 7}
\begin{itemize}
    \item {\textbf{Worked On}}
    \begin{itemize}
        
    \end{itemize}

    \item {\textbf{Problems Encountered}}
    \begin{itemize}
        
    \end{itemize}

    \item{\textbf{Plans for next week}}
    \begin{itemize}
        
    \end{itemize}

\end{itemize}

\subsubsection{Spring Week 8}
\begin{itemize}
    \item {\textbf{Worked On}}
    \begin{itemize}
        
    \end{itemize}

    \item {\textbf{Problems Encountered}}
    \begin{itemize}
        
    \end{itemize}

    \item{\textbf{Plans for next week}}
    \begin{itemize}
        
    \end{itemize}

\end{itemize}

\subsection{Tao's Blog Entries}

\subsubsection{Fall Week 4}
\begin{itemize}
    \item {\textbf{Worked On}}
    \begin{itemize}
        
    \end{itemize}

    \item {\textbf{Problems Encountered}}
    \begin{itemize}
        
    \end{itemize}

    \item{\textbf{Plans for next week}}
    \begin{itemize}
        
    \end{itemize}

\end{itemize}

\subsubsection{Fall Week 5}
\begin{itemize}
    \item {\textbf{Worked On}}
    \begin{itemize}
        
    \end{itemize}

    \item {\textbf{Problems Encountered}}
    \begin{itemize}
        
    \end{itemize}

    \item{\textbf{Plans for next week}}
    \begin{itemize}
        
    \end{itemize}

\end{itemize}

\subsubsection{Fall Week 6}
\begin{itemize}
    \item {\textbf{Worked On}}
    \begin{itemize}
        
    \end{itemize}

    \item {\textbf{Problems Encountered}}
    \begin{itemize}
        
    \end{itemize}

    \item{\textbf{Plans for next week}}
    \begin{itemize}
        
    \end{itemize}

\end{itemize}

\subsubsection{Fall Week 7}
\begin{itemize}
    \item {\textbf{Worked On}}
    \begin{itemize}
        
    \end{itemize}

    \item {\textbf{Problems Encountered}}
    \begin{itemize}
        
    \end{itemize}

    \item{\textbf{Plans for next week}}
    \begin{itemize}
        
    \end{itemize}

\end{itemize}

\subsubsection{Fall Week 8}
\begin{itemize}
    \item {\textbf{Worked On}}
    \begin{itemize}
        
    \end{itemize}

    \item {\textbf{Problems Encountered}}
    \begin{itemize}
        
    \end{itemize}

    \item{\textbf{Plans for next week}}
    \begin{itemize}
        
    \end{itemize}

\end{itemize}

\subsubsection{Fall Week 9}
\begin{itemize}
    \item {\textbf{Worked On}}
    \begin{itemize}
        
    \end{itemize}

    \item {\textbf{Problems Encountered}}
    \begin{itemize}
        
    \end{itemize}

    \item{\textbf{Plans for next week}}
    \begin{itemize}
        
    \end{itemize}

\end{itemize}

\subsubsection{Fall Week 10}
\begin{itemize}
    \item {\textbf{Worked On}}
    \begin{itemize}
        
    \end{itemize}

    \item {\textbf{Problems Encountered}}
    \begin{itemize}
        
    \end{itemize}

    \item{\textbf{Plans for next week}}
    \begin{itemize}
        
    \end{itemize}

\end{itemize}

\subsubsection{Fall Week 11}
\begin{itemize}
    \item {\textbf{Worked On}}
    \begin{itemize}
        
    \end{itemize}

    \item {\textbf{Problems Encountered}}
    \begin{itemize}
        
    \end{itemize}

    \item{\textbf{Plans for next week}}
    \begin{itemize}
        
    \end{itemize}

\end{itemize}

\subsubsection{Winter Week 1}
\begin{itemize}
    \item {\textbf{Worked On}}
    \begin{itemize}
        
    \end{itemize}

    \item {\textbf{Problems Encountered}}
    \begin{itemize}
        
    \end{itemize}

    \item{\textbf{Plans for next week}}
    \begin{itemize}
        
    \end{itemize}

\end{itemize}

\subsubsection{Winter Week 2}
\begin{itemize}
    \item {\textbf{Worked On}}
    \begin{itemize}
        
    \end{itemize}

    \item {\textbf{Problems Encountered}}
    \begin{itemize}
        
    \end{itemize}

    \item{\textbf{Plans for next week}}
    \begin{itemize}
        
    \end{itemize}

\end{itemize}

\subsubsection{Winter Week 3}
\begin{itemize}
    \item {\textbf{Worked On}}
    \begin{itemize}
        
    \end{itemize}

    \item {\textbf{Problems Encountered}}
    \begin{itemize}
        
    \end{itemize}

    \item{\textbf{Plans for next week}}
    \begin{itemize}
        
    \end{itemize}

\end{itemize}

\subsubsection{Winter Week 4}
\begin{itemize}
    \item {\textbf{Worked On}}
    \begin{itemize}
        
    \end{itemize}

    \item {\textbf{Problems Encountered}}
    \begin{itemize}
        
    \end{itemize}

    \item{\textbf{Plans for next week}}
    \begin{itemize}
        
    \end{itemize}

\end{itemize}

\subsubsection{Winter Week 5}
\begin{itemize}
    \item {\textbf{Worked On}}
    \begin{itemize}
        
    \end{itemize}

    \item {\textbf{Problems Encountered}}
    \begin{itemize}
        
    \end{itemize}

    \item{\textbf{Plans for next week}}
    \begin{itemize}
        
    \end{itemize}

\end{itemize}

\subsubsection{Winter Week 6}
\begin{itemize}
    \item {\textbf{Worked On}}
    \begin{itemize}
        
    \end{itemize}

    \item {\textbf{Problems Encountered}}
    \begin{itemize}
        
    \end{itemize}

    \item{\textbf{Plans for next week}}
    \begin{itemize}
        
    \end{itemize}

\end{itemize}

\subsubsection{Winter Week 7}
\begin{itemize}
    \item {\textbf{Worked On}}
    \begin{itemize}
        
    \end{itemize}

    \item {\textbf{Problems Encountered}}
    \begin{itemize}
        
    \end{itemize}

    \item{\textbf{Plans for next week}}
    \begin{itemize}
        
    \end{itemize}

\end{itemize}

\subsubsection{Winter Week 8}
\begin{itemize}
    \item {\textbf{Worked On}}
    \begin{itemize}
        
    \end{itemize}

    \item {\textbf{Problems Encountered}}
    \begin{itemize}
        
    \end{itemize}

    \item{\textbf{Plans for next week}}
    \begin{itemize}
        
    \end{itemize}

\end{itemize}

\subsubsection{Winter Week 9}
\begin{itemize}
    \item {\textbf{Worked On}}
    \begin{itemize}
        
    \end{itemize}

    \item {\textbf{Problems Encountered}}
    \begin{itemize}
        
    \end{itemize}

    \item{\textbf{Plans for next week}}
    \begin{itemize}
        
    \end{itemize}

\end{itemize}

\subsubsection{Winter Week 10}
\begin{itemize}
    \item {\textbf{Worked On}}
    \begin{itemize}
        
    \end{itemize}

    \item {\textbf{Problems Encountered}}
    \begin{itemize}
        
    \end{itemize}

    \item{\textbf{Plans for next week}}
    \begin{itemize}
        
    \end{itemize}

\end{itemize}

\subsubsection{Winter Week 11}
\begin{itemize}
    \item {\textbf{Worked On}}
    \begin{itemize}
        
    \end{itemize}

    \item {\textbf{Problems Encountered}}
    \begin{itemize}
        
    \end{itemize}

    \item{\textbf{Plans for next week}}
    \begin{itemize}
        
    \end{itemize}

\end{itemize}

\subsubsection{Spring Week 1}
\begin{itemize}
    \item {\textbf{Worked On}}
    \begin{itemize}
        
    \end{itemize}

    \item {\textbf{Problems Encountered}}
    \begin{itemize}
        
    \end{itemize}

    \item{\textbf{Plans for next week}}
    \begin{itemize}
        
    \end{itemize}

\end{itemize}

\subsubsection{Spring Week 2}
\begin{itemize}
    \item {\textbf{Worked On}}
    \begin{itemize}
        
    \end{itemize}

    \item {\textbf{Problems Encountered}}
    \begin{itemize}
        
    \end{itemize}

    \item{\textbf{Plans for next week}}
    \begin{itemize}
        
    \end{itemize}

\end{itemize}

\subsubsection{Spring Week 3}
\begin{itemize}
    \item {\textbf{Worked On}}
    \begin{itemize}
        
    \end{itemize}

    \item {\textbf{Problems Encountered}}
    \begin{itemize}
        
    \end{itemize}

    \item{\textbf{Plans for next week}}
    \begin{itemize}
        
    \end{itemize}

\end{itemize}

\subsubsection{Spring Week 4}
\begin{itemize}
    \item {\textbf{Worked On}}
    \begin{itemize}
        
    \end{itemize}

    \item {\textbf{Problems Encountered}}
    \begin{itemize}
        
    \end{itemize}

    \item{\textbf{Plans for next week}}
    \begin{itemize}
        
    \end{itemize}

\end{itemize}

\subsubsection{Spring Week 5}
\begin{itemize}
    \item {\textbf{Worked On}}
    \begin{itemize}
        
    \end{itemize}

    \item {\textbf{Problems Encountered}}
    \begin{itemize}
        
    \end{itemize}

    \item{\textbf{Plans for next week}}
    \begin{itemize}
        
    \end{itemize}

\end{itemize}

\subsubsection{Spring Week 6}
\begin{itemize}
    \item {\textbf{Worked On}}
    \begin{itemize}
        
    \end{itemize}

    \item {\textbf{Problems Encountered}}
    \begin{itemize}
        
    \end{itemize}

    \item{\textbf{Plans for next week}}
    \begin{itemize}
        
    \end{itemize}

\end{itemize}

\subsubsection{Spring Week 7}
\begin{itemize}
    \item {\textbf{Worked On}}
    \begin{itemize}
        
    \end{itemize}

    \item {\textbf{Problems Encountered}}
    \begin{itemize}
        
    \end{itemize}

    \item{\textbf{Plans for next week}}
    \begin{itemize}
        
    \end{itemize}

\end{itemize}

\subsubsection{Spring Week 8}
\begin{itemize}
    \item {\textbf{Worked On}}
    \begin{itemize}
        
    \end{itemize}

    \item {\textbf{Problems Encountered}}
    \begin{itemize}
        
    \end{itemize}

    \item{\textbf{Plans for next week}}
    \begin{itemize}
        
    \end{itemize}

\end{itemize}

\subsection{Dan's Blog Entries}

\subsubsection{Fall Week 4}
\begin{itemize}
    \item {\textbf{Worked On}}
    \begin{itemize}
        \item Created the problem statement tex template.
        \item The problem statement proposed solution section.        
    \end{itemize}

    \item {\textbf{Problems Encountered}}
    \begin{itemize}
        \item Not getting feedback on the revised problem statement as quickly as I would have liked.
    \end{itemize}
    \item{\textbf{Plans for next week}}
    \begin{itemize}
        \item Create LaTeX template for the SRS.
        \item Research LaTeX Gantt charts.
        \item Start on SRS.
        \item Start on filling out a Gantt chart.        
    \end{itemize}


\end{itemize}

\subsubsection{Fall Week 5}
\begin{itemize}
    \item {\textbf{Worked On}}
    \begin{itemize}
        \item Created an SRS tex template.
        \item Did not use IEEEtran.cls because it does not format the SRS properly.
        \item I found a template online that has the proper formatting and modified it to follow the IEEE 830 documentation.
        \item https://github.com/Eisenbarth/SRS-Tex
        \item Created a new 'srs' git branch.
    
        \item Has a new 'srs\_template' folder with the template tex file, makefile, and supporting resources.
        \item Worked on the SRS.
    
        \item Filled out Software Interfaces and Communications interfaces.
        \item Found templates for LaTeX Gantt charts.        
    \end{itemize}

    \item {\textbf{Problems Encountered}}
    \begin{itemize}
        \item Trying to write a document that assumes we know what components we will need and will be using without really knowing what those things are.        
    \end{itemize}

    \item{\textbf{Plans for next week}}
    \begin{itemize}
        \item Work on SRS final draft.
        \item Work on / finish Gantt chart.
        \item Start thinking about the tech review.
    \end{itemize}

\end{itemize}

\subsubsection{Fall Week 6}
\begin{itemize}
    \item {\textbf{Worked On}}
    \begin{itemize}
        \item Corrected our SRS tex document to use IEEEtran.cls.
        \item Corrected the heirarchy format to be numeric, instead of the default Roman numeral.
        \item Gantt chart:
        \begin{itemize}
            \item Created LaTeX document for rendering a Gantt chart:
            \item Added files and folder for gantt chart to repo
            \item Added comments explaining how to set up groups and tasks.
            \item Filled out SRS group with subtasks.
            \item Created system for groups to track subtask progress.
            \item Fixed formatting.
            \item Finished basic layout for Gantt chart
            \item Integrated gantt\_chart.tex into arc\_srs\_template.tex
        \end{itemize}
    \end{itemize}

    \item {\textbf{Problems Encountered}}
    \begin{itemize}
        \item Getting the Gantt charts to render in LaTeX. It took around 9 hours, and it's not even close to scale-able.
    \end{itemize}

    \item{\textbf{Plans for next week}}
    \begin{itemize}
        \item Work on finishing SRS final draft.
        \item Determine what my part of the tech review is.
        \item Work on tech review.
    \end{itemize}
\end{itemize}

\subsubsection{Fall Week 7}
\begin{itemize}
    \item {\textbf{Worked On}}
    \begin{itemize}
        \item Tech Review: Researched image analysis software, telemetry radios, and user interfaces for drones.
        \begin{itemize}
            \item Image analysis software
            \item Found DroneKit-Python, ArduPilot, and LibrePilot.
            \item Telemetry radios
            \item Found 3DR 915 MHz Transceiver, RFD900 Radio Modem, and OpenPilot OPLink Mini Ground and Air Station 433 MHz
            \item User interfaces
            \item Found QGroundControl, DroneKit-Android, and LibrePilot.
            \item Worked on the write up for these components.
        \end{itemize}
    \end{itemize}

    \item {\textbf{Problems Encountered}}
    \begin{itemize}
        \item The documentation for the drone software is rather vague when it comes to information on path-finding and image analysis capabilities.
        \item DroneKit documentation was somewhat confusing. At times it seemed like it was its own project, but at other times it referenced ArduPilot, making me think that DroneKit is a Python API wrapper on top of ArduPilot.
    \end{itemize}

    \item{\textbf{Plans for next week}}
    \begin{itemize}
    \item Finish SRS document.
    \item Finish Tech Review document.
    \item Start:
    \begin{itemize}
        \item Planning the design document.
        \item Planning the progress report and presentation.
        \item Planning the poster.    
    \end{itemize}
    \end{itemize}
\end{itemize}

\subsubsection{Fall Week 8}
\begin{itemize}
    \item {\textbf{Worked On}}
    \begin{itemize}
        \item Completing tech review and additions to the srs document.
        \item Researched hardware for telemetry radios.
        \item Researched software for UI to be able to communicate with the vehicle.
        \item Researched software for image analysis for environment mapping and depth finding based off of sensor data.
    \end{itemize}

    \item {\textbf{Problems Encountered}}
    \begin{itemize}
        \item Not too much different that other weeks: the document formats require the writer to have concrete examples and plans for where the project is going. Since this is a research project, and has not been done before, our work-flow really requires some experimentation before we can establish what is going to be used.
    \end{itemize}

    \item{\textbf{Plans for next week}}
    \begin{itemize}
        \item Work on creating a plan that will be used in the Design Document.
        \item Create the design document.
    \end{itemize}
\end{itemize}

\subsubsection{Fall Week 9}
\begin{itemize}
    \item {\textbf{Worked On}}
    \begin{itemize}
        \item Filled out specific requirements section of the SRS.
        \item Wrote "Radio Communications", "Image Analysis", and "User Interfaces" sections of the the tech review.
    \end{itemize}

    \item {\textbf{Problems Encountered}}
    \begin{itemize}
        \item Having been delayed on the previous documents, we are about 1 to 1.5 weeks behind schedule. We need to finish up the srs and tech review still. This makes starting on the design document difficult, if not impossible.
    \end{itemize}

    \item{\textbf{Plans for next week}}
    \begin{itemize}
        \item Write the design document.
    \end{itemize}

\end{itemize}

\subsubsection{Fall Week 10}
\begin{itemize}
    \item {\textbf{Worked On}}
    \begin{itemize}
        \item Finishing the srs and design documents.
        \item Wrote up how we will test our implementation of the ARC project.  
        My part of the progress report is to synthesize my comments in these progress posts into a coherent subsection of the report.  
        For the presentation, I provided summaries on power point slides, these included graphics and diagrams.
    \end{itemize}

    \item {\textbf{Problems Encountered}}
    \begin{itemize}
        \item Our implementation for many components of ARC is currently unknown. More time is required to know exactly what specific software will be used on our project. We need to dive into the APIs used and see how compatible they are and see if we can write our own limited API (meaning simple conversions). So far we have only really had time to look into what open source resources are available at a higher level. This makes writing a detailed design document (in terms of actual code implementation) difficult. So, the design document is being written in terms of experimental plan. We have milestones that our project needs to meet. After succeeding in the first milestone, we continue to the next.
    \end{itemize}

    \item{\textbf{Plans for next week}}
    \begin{itemize}
        \item Write up my section of the progress report.
        \item Record the progress report presentation.


    \end{itemize}
\end{itemize}

\subsubsection{Fall Week 11}
\begin{itemize}
    \item {\textbf{Worked On}}
    \begin{itemize}
        \item Progress Report  
        Created LaTeX template.
        Wrote the purpose and goals, weekly summaries for weeks 1-6.
    \end{itemize}

    \item {\textbf{Problems Encountered}}
    \begin{itemize}
        \item It's finals week, enough said.
    \end{itemize}

    \item{\textbf{Plans for next week}}
    \begin{itemize}
        \item It's Christmas break, but I plan to start looking into implementation, how the APIs will work.
    \end{itemize}
\end{itemize}

\subsubsection{Winter Week 1}
\begin{itemize}
    \item {\textbf{Worked On}}
    \begin{itemize}
        \item Setting up the development environment for Linux and ROS.
        \begin{itemize}
            \item We will be using the Robotics Operating System (ROS) for communication and control of our vehicle. Linux is the operating system of choice for the ROS platform.
            \item I worked through orientation tutorials for ROS to become more familiar with the platform.
        \end{itemize}
    \end{itemize}

    \item {\textbf{Problems Encountered}}
    \begin{itemize}
        \item No real problems encountered in week 1.
    \end{itemize}

    \item{\textbf{Plans for next week}}
    \begin{itemize}
        \item Get Linux installed on my laptop to dual boot, also get Linux running in VMware for further testing/development capabilities.
    \end{itemize}
\end{itemize}

\subsubsection{Winter Week 2}
\begin{itemize}
    \item {\textbf{Worked On}}
    \begin{itemize}
        \item Installed Linux to dual boot on my laptop.
        \item Got ROS running in Ubuntu 14.04 and 16.04.
        \item Got GT Autorally working in 14.04 and 16.04 in both "native" Ubuntu and in a VM.
    \end{itemize}

    \item {\textbf{Problems Encountered}}
    \begin{itemize}
        \item My laptop died on Tuesday, so I ordered a new one and got it set up on Thursday/Friday.
    \end{itemize}

    \item{\textbf{Plans for next week}}
    \begin{itemize}
        \item Start digging into how Autorally uses ROS and Gazebo (a simulator) to see how we can use it in our project.
        \item Look into direct computer-to-computer communication.
    \end{itemize}
\end{itemize}

\subsubsection{Winter Week 3}
\begin{itemize}
    \item {\textbf{Worked On}}
    \begin{itemize}
        \item Continued with development environment setup.
        \item More work in ROS.
        \item Found software library for our IMU.
        \item Started looking into computer-to-computer direct communication.
    \end{itemize}

    \item {\textbf{Problems Encountered}}
    \begin{itemize}
        \item The research I have done stated that new computer should be able to do direct ethernet communication without a crossover cable. I was not able to get the computers to communicate. I need to get a crossover cable to see if that is the issue, or if some other setting is incorrect.
        \item We received a corrupt image for the PXFMini, so we will need a new one sent to us.
        \item We were hoping we could find a ready-made computer model for our RC car for ROS or Gazebo, we have not found one as yet. This means that we might need to spend quite a bit more time building a model for simulation, if we want to go that route for development.
    \end{itemize}

    \item{\textbf{Plans for next week}}
    \begin{itemize}
        \item Get computers talking to each other directly over ethernet.
        \item Get telemetry data from the RC to a remote computer.
        \item Get command from remote computer to RC car.
    \end{itemize}
\end{itemize}

\subsubsection{Winter Week 4}
\begin{itemize}
    \item {\textbf{Worked On}}
    \begin{itemize}
        \item Got computers talking with each other directly over ethernet.

        \item In Ubuntu on both systems:

Edit wired connection
set ipv4 to manual
set the IP address
subnet mask, and gateway
The networking service may need to be started:
sudo service network-manager restart
Got Raspberry Pi3 talking with computers over ethernet as well.

        \item Worked on tutorials for ROS publishers and subscribers

        \item Got ROS publisher on computer A talking with ROS subscriber on computer B.

        \item This test is between my personal laptop and the Raspberry Pi 3.
        \begin{itemize}
        \item It is important that ~/.bashrc have the following lines:\\
        For computer running ROS master (this is the Raspberry Pi, and note that IP addresses are the same):
        export export ROS\_MASTER\_URI=http://192.168.1.2:11311
        export ROS\_IP=192.168.1.2
        For client computer (note ROS\_IP is local IP address):
        export ROS\_MASTER\_URI=http://192.168.1.2:11311
        export ROS\_IP=192.168.1.3            
        \end{itemize}
        \item Got the NUC talking to the Raspberry Pi 3 in ROS following a publisher/subscriber tutorial:\\
        http://docs.erlerobotics.com/robot\_operating\_system/ros/basic\_concepts/examples/publisher\_and\_subscribers

        \item Able to get internet working over ethernet. This will allow us to update the Raspberry Pi3, install needed ROS packages and other library dependencies.
    \end{itemize}

    \item {\textbf{Problems Encountered}}
    \begin{itemize}
        \item The keyboard layout on the Raspberry Pi 3 was set to UK by default.
        \item Fixed the layout permanently with: sudo raspi-config
        \item Other settings can be configured for the Pi 3 from the same config menu.
        \item WiFi is not working on the Raspberry Pi 3, internet is not working, in general.
        \item Able to get internet working over ethernet.
        \item The PXFMini did not power on.\\
        Discovered that the PXFMini cape was connected to the Raspberry Pi 3 incorrectly.\\
        After connecting correctly, the Rpi3 boots, but the PXFMini LED status lights show a code that tells us that it could not launch. This could be normal, or caused by an error. The documentation says that a vehicle selection should appear when the Rpi3 boots up, but I do not see that.
        \item Upgraded the Raspberry Pi 3 using sudo apt-get dist-upgrade and then the rpi would not boot any more.
        \begin{itemize}
            \item It turns out that the upgrade somehow wiped out the required kernel7.img needed for the OS.
            \item Copied a kernel7.img from another system image, along with other missing files, the rpi now boots, and WiFi is now working too!
        \end{itemize}
        \item Discovered that our power supply (USB power from the NUC) is under-powered for the Raspberry Pi 3, especially with the PXFMini attached. We will need to try using the DC power converter and the passthrough to the Rpi. We will need to get a voltage-meter to make sure our power is correct (need 5V, 2.5A).
    \end{itemize}

    \item{\textbf{Plans for next week}}
    \begin{itemize}
        \item Get proper power to the Raspberry Pi 3.
        \item Get the PXFMini runnning properly.
        \item Make servos and motors run via the ROS and the PXFMini.
    \end{itemize}
\end{itemize}

\subsubsection{Winter Week 5}
\begin{itemize}
    \item {\textbf{\textbf{Worked On}}}
    \begin{itemize}
        \item Got the PXFmini launched and armed with instructions for installing binaries for APMrover2.
        \item Got the Raspberry Pi 3 (RPi3) connected to the internet via ethernet using the original Erle OS image.
        \begin{itemize}
            \item Steps for connecting to internet using ethernet:\\
            In /etc/network/interfaces, uncomment the eth0 section.
            \item Steps for getting WiFi internet working with home router you will need to edit two files, interfaces and wpa\_supplicant.conf:
            \begin{itemize}
                \item Back up original /etc/network/interfaces file.
                \item In /etc/network/interfaces, comment out the lines for wlan0 and add the following lines
                \begin{itemize}
                    \item auto wlan0
                    \item iface wlan0 inet dhcp
                    \item netmask 255.255.255.0
                    \item wpa-conf /etc/wpa\_supplicant/wpa\_supplicant.conf
                \end{itemize}
                \item Back up original /etc/wpa\_supplicant/wpa\_supplicant.conf
                \begin{itemize}
                    \item Edit /etc/wpa\_supplicant/wpa\_supplicant.conf:
                    \item Comment out current network block
                    \item Add new network={...} block containing the following lines:
                    \item ssid="<name-of-network-ssid>"
                    \item psk="<network-password>"
                \end{itemize}

                \item Reboot the Rpi3.
            \end{itemize}



        \item I was able to publish to mavros topics and confirmed that the subscribers did get the information. So far it has not caused the actuators to fire.
        
        \item I was also able to confirm that the pxfmini is talking with the rpi, I can see the imu data being transmitted and changing when I move the unit around.
        \end{itemize}
    \end{itemize}

    \item {\textbf{Problems Encountered}}
    \begin{itemize}
        \item Continued to struggle with getting internet working using WiFi on the RPi3 and the original Erle image. This is important to get the proper software installed.

        \item I can't compile any of the code that is given in the examples because dependencies are missing... I've gotten one script to actually run, but it doesn't do anything for the pxf. and the documentation is so incomplete, it just assumes that "everything" (whatever that is) is set up and ready to go to launch the given script. And in places where it tells you to do something, it doesn't explain what is necessary for it to run, or even what the code is doing. When I try to compile a cpp file (ground\_rover.cpp) with include<ros/ros.h> the g++ compiler says it can't find the file, which means that the ros library wasn't installed into the system include directory. I can compile using an include option and point to the correct directory for the ros include libraries, but then it errors out on a different file not found in one of the included headers of ground\_rover.cpp.

        \item The instructions from erle-robotics in "first steps" show a mavros node 'raspicam\_node when running rosnode list, that node is not listed after going through their instructions. If I run rostopic list I get a similar output, but rostopics and rosnodes are two different things.

        \item Getting WiFi working for the internet breaks launching rostopics. The wifi settings must be set back to original to restore rostopic settings. -- Use erle-reset.

        \item After talking with our client, Kevin McGrath, we have determined that something is probably not set up properly, or strangery of some kind is afoot. The rpi3 and Erle PXFmini should work with the erle-image "out of the box", but it does not on our rpi3, for an as-of-yet unknown reason. Kevin and I are going to try to figure out what is going on.

        \item Well, I am now able to publish (via command line) to any of the mavros topic subscribers. I have only published to few of them, but I know how to do it now. I am also able to subscribe to topics to see what they are sending. The pxfmini is "working", in the sense that is communicating with the rpi3. I can see the IMU data, which changes when I move the pxfmini. I don't know if the numbers or correct, or even usable, but at least they are numbers...

        \item Unfortunately, nothing I have tried has gotten the actuator to move. When I arm the pxfmini, the actuator fires erratically until I send a message via mavros/rc/override, then the motor stops, but further messages do not cause the motor to do anything.\\
        At this point, I feel like I could spend the rest of the term figuring this out. Obviously that is not an option.\\
        I'm talking with Kevin about this problem. He's also having a hard time getting the pxfmini to work via software.

        \item Progress is going very slowly due to very poor documentation for the pxfmini and mavros (an api for RC communication).
    \end{itemize}

    \item{\textbf{Plans for next week}}
    \begin{itemize}
        \item It's week 6 and the midterm report and presentation is due.
        \item Get our required One Note project environment setup and filled out.
        \item Make revisions to the documentation for my areas of focus.
        \item Make my portion of the video presentation.
    \end{itemize}
\end{itemize}

\subsubsection{Winter Week 6}
\begin{itemize}
    \item {\textbf{Worked On}}
    \begin{itemize}
        \item Got the OneNote workbook up.
        \item Added document and presentation PDFs to OneNote.
        \item Added revision table templates to OneNote
        \item Added "template" or empty revision tables to documents (Problem Statement, Tech Review, SRS, Design).
        \item Made revisions to:
        \begin{itemize}
            \item tech review
            \item design doc
            \item Looked at srs and updated gantt chart.
            \item Created template for winter midterm report.
        \end{itemize}
    \end{itemize}

    \item {\textbf{Problems Encountered}}
    \begin{itemize}
        \item None.
    \end{itemize}

    \item{\textbf{Plans for next week}}
    \begin{itemize}
        \item Install QGroundControl on my system and see if we can control the pxfmini.
        \item Try to get "gtest" (a library necessary for one of the erle tutorials) installed.
        \item make a determination if we want to continue with trying to use the pxfmini autopilot or switch to a simple controller.
    \end{itemize}
\end{itemize}

\subsubsection{Winter Week 7}
\begin{itemize}
    \item {\textbf{Worked On}}
    \begin{itemize}
        \item Got the "Teleoperating" tutorial installed. The tutorials allows control of the RC vehicle using arrow keys on the keyboard. Kevin McGrath figured out how to install it:
        \begin{itemize}
            \item mkdir -p ~/erle\_ws/src
            \item cd ~/erle\_ws/src
            \item git clone https://github.com/erlerobot/gazebo\_cpp\_examples
            \item git clone https://github.com/ros-perception/vision\_opencv
            \item git clone https://github.com/ros-perception/image\_common
            \item cd ..
            \item cd /usr/lib/arm-linux-gnueabihf/
            \item sudo ln -s libboost\_python-py34.so libboost\_python3.so
            \item sudo apt-get install libgtest-dev
            \item catkin\_make --pkg ros\_erle\_cpp\_teleoperation\_erle\_rover
        \end{itemize}

        \item The package now installs, and the keyboard changes the input to the rostopic /mavros/rc/override, but the motor does not activate.

        \item Able to set the parameter SYSID\_MYGCS, this is needed to override RC in and enable computer control of the autopilot.
        \begin{itemize}
            \item To set using \$ rosrun mavros mavparam set SYSID\_MYGCS 1, the autopilot must be launched and armed.
        \end{itemize}
        \item Able to use /mavros/rc/override to publish data to /mavros/rc/in. This is very important because it is necessary for controlling motors. The motors are still not moving, however.

        \item Installed GCS software (ArduPilot) and connected telemetry radios. Now able to see attitude of the autopilot in a gimble. But nothing else is connecting, not able to control motors at all. Cannot disarm/arm the autopilot from GCS.
    \end{itemize}

    \item {\textbf{Problems Encountered}}
    \begin{itemize}
        \item Using the above "Teleoperation" tutorial, I am able to change the channel values for steering and throttle to /mavros/rc/in. I can see these values change on the GCS via the telemetry radios. This means that data is being sent over mavros and the telemetry radios are communicating properly. So, the problem is somewhere between /mavros/rc/in and the PXFmini. At this point we have spent almost 4 weeks trying to get the PXFmini to work. This was supposed to work "out of the box". We will move on to a different autopilot.
    \end{itemize}

    \item{\textbf{Plans for next week}}
    \begin{itemize}
        \item Work out the kinks while installing AutoRally.
    \end{itemize}
\end{itemize}

\subsubsection{Winter Week 8}
\begin{itemize}
    \item {\textbf{Worked On}}
    \begin{itemize}
        \item Went through uninstalling and reinstalling Autorally.
        \item The autorally softare package has a few different parts to it and odd/obscure errors can occur during installation. My goal was to find errors that arise and document how to fix/correct them so that users will have a better installation experience and get the system up and running faster.

        \item Found the following error:
\begin{verbatim}
    CMake Error at autorally/autorally\_control/CMakeLists.txt:19 (find_package): By not providing "FindEigen3.cmake" in CMAKE_MODULE_PATH this project has asked CMake to find a package configuration file provided by "Eigen3", but CMake did not find one.

    Could not find a package configuration file provided by "Eigen3" with any of the following names:

    Eigen3Config.cmake
    eigen3-config.cmake
    
    Add the installation prefix of "Eigen3" to CMAKE_PREFIX_PATH or set "Eigen3_DIR" to a directory containing one of the above files. If "Eigen3" provides a separate development package or SDK, be sure it has been installed.
\end{verbatim}

    \item This was resolved by changing the calling CMakeLists.txt file:\\
    Original: find\_package(Eigen3 REQUIRED)\\
    Changed to:\\
    find\_package(PkgConfig)\\
    pkg\_search\_module(Eigen3 REQUIRED eigen3)\\
    
    \item I was able to reproduce the above error and verify the above fix worked consistently. I have not been able to test this on other systems.
    
    \item Worked through the uninstall process and identified where files were install to, this mostly had to do with GTSAM installation. Autorally itself is fairly localized and doesn't install any libraries to /usr/..
    
    \item I have a fairly solid grasp of the uninstall/reinstall process for autorally, which means that as we move forward with integrating our own software we will be able to provide a more smooth experience for other users through documentation, and perhaps automation of the installation process.
    
    \item Still not sure if the Eigen3 error is something that will occur on a first-time install. I don't recall seeing the error before.
        
    \end{itemize}

    \item {\textbf{Problems Encountered}}
    \begin{itemize}
        \item It took a few days to figure out what all needed to be cleaned out of the system to be able to reinstall the software. Then it took another day or two to figure out how to resolve installation errors that we hadn't seen before.
    \end{itemize}

    \item{\textbf{Plans for next week}}
    \begin{itemize}
        \item         \item Get autorally installed with the new files Tao is working on.
        \item Get Telemetry radios working without the pxfmini...
    \end{itemize}
\end{itemize}

\subsubsection{Winter Week 9}
\begin{itemize}
    \item {\textbf{Worked On}}
    \begin{itemize}
        \item Integrating Tao's implementation in to the installation process.
        \item Able to launch rviz using a custom configuration file. This allows us to load a specific configuration on different computers which will eliminate the need for tedious setup of the rviz environment.
    \end{itemize}

    \item {\textbf{Problems Encountered}}
    \begin{itemize}
        \item None this week.
    \end{itemize}

    \item{\textbf{Plans for next week}}
    \begin{itemize}
        \item Continue with building our installation process.
    \end{itemize}
\end{itemize}

\subsubsection{Winter Week 10}
\begin{itemize}
    \item {\textbf{Worked On}}
    \begin{itemize}
        \item None
    \end{itemize}

    \item {\textbf{Problems Encountered}}
    \begin{itemize}
        \item My wife had a pregnancy-related emergency and our son was born via emergency c-section.
    \end{itemize}

    \item{\textbf{Plans for next week}}
    \begin{itemize}
        \item None.
    \end{itemize}
\end{itemize}

\subsubsection{Winter Week 11}
\begin{itemize}
    \item {\textbf{Worked On}}
    \begin{itemize}
        \item None.
    \end{itemize}

    \item {\textbf{Problems Encountered}}
    \begin{itemize}
        \item I was in the NICU in Eugene.
    \end{itemize}

    \item{\textbf{Plans for next week}}
    \begin{itemize}
        \item None -- Spring Break
    \end{itemize}
\end{itemize}

\subsubsection{Spring Week 1}
\begin{itemize}
    \item {\textbf{Worked On}}
    \begin{itemize}
        \item Talked through the complexity of AutoRally with Tao and Cierra. We determined that AutoRally is too complex to try to integrate into our project in the time-frame we have left. We decided to drop AutoRally and focus solely on ROS for the navigation stack.
    \end{itemize}

    \item {\textbf{Problems Encountered}}
    \begin{itemize}
        \item The physical USB port on one of your telemetry radios broke. We cannot do radio communication until it is fixed.
    \end{itemize}

    \item{\textbf{Plans for next week}}
    \begin{itemize}
        \item Get the IMU, Lidar, Leddar, and GPS drivers installed on the NUC.
    \end{itemize}
\end{itemize}

\subsubsection{Spring Week 2}
\begin{itemize}
    \item {\textbf{Worked On}}
    \begin{itemize}
        \item Installed the IMU and Lidar drivers on the NUC.
        \item Got rostopics for the IMU and Lidar publishing data and able to visualize the data in RViz.
        \item Worked with Tao on figuring out how to get another simulation (Stage) working with our hardware packages.
    \end{itemize}

    \item {\textbf{Problems Encountered}}
    \begin{itemize}
        \item The Leddar unit is not being recognized in Windows or Linux via USB. It was working earlier, but isn't now. We will try the RS485 interface to see if the Leddar unit is bad, or just the USB interface.
    \end{itemize}

    \item{\textbf{Plans for next week}}
    \begin{itemize}
        \item Integrate IMU and GPS.
        \item Get simulation stable using GPS to smooth out the IMU calculations.
        \item Refine hardware mounting on the car
    \end{itemize}
\end{itemize}

\subsubsection{Spring Week 3}
\begin{itemize}
    \item {\textbf{Worked On}}
    \begin{itemize}
        \item Edited the outcomes and findings for the poster.
        \item Continued to try to get IMU and GPS data converted into something the AutoRally state estimator will accept.
    \end{itemize}

    \item {\textbf{Problems Encountered}}
    \begin{itemize}
        \item Still having no luck converting the data we need.
    \end{itemize}

    \item{\textbf{Plans for next week}}
    \begin{itemize}
        \item Help Tao with the simulations and Cierra with the hardware mounting.
    \end{itemize}
\end{itemize}

\subsubsection{Spring Week 4}

\begin{itemize}
    \item {\textbf{Worked On}}
    \begin{itemize}
        \item Attempting to convert our ROS-published imu data to something that imuGPSestimator can use. imuGPSestimator is from GT AutoRally, we need it to integrate our GPS and IMU data to determine the position of the car in the world.
    \end{itemize}

    \item {\textbf{Problems Encountered}}
    \begin{itemize}
        \item We need the imu and gps data to accurately track the position of our car in the world. Integrating that data is not trivial. This is absolutely required to do any autonomous navigation in the real world. It turns out that this very problem is what was really helpful in having an autopilot. The autopilot does all this integration for you. Without the autopilot we have to come up with an integration and interpretation solution on our own.
\item We do not have time to develop our own software to integrate GPS and IMU data to accurately track the position of our car in the world.
    \begin{itemize}
        \item We are trying to use the estimator from GT AutoRally to integrate GPS and IMU data. However, imuGPSestimator subscribes to a different data type (IMUConstPtr) than the data type published by by our imu (IMU data type).
        \item We are trying to resolve this difference by converting IMU to IMUConstPtr. This has proven very challenging. There is no documentation on what IMUConstPtr actually is. The definition shows us that it's super type is shared\_ptr from the Boost library.
        \item shared\_ptr seems to be a generic type used to track pointer usage and garbage collect automatically when the pointer is no longer used. It does not seem to be relevant to our need to convert from IMU to IMUConstPtr.
    \end{itemize}

    \item So far, we have not been able to convert the data.
    \end{itemize}

    \item{\textbf{Plans for next week}}
    \begin{itemize}
        \item Do as much as we can to get our software running the car.
        \item Get the poster more-or-less finalized and approved by our client.
        \item Get all the code we have been working on/with uploaded to our Github repo.
        \item Submit our poster for print before May 1.
    \end{itemize}
\end{itemize}

\subsubsection{Spring Week 5}
\begin{itemize}
    \item {\textbf{Worked On}}
    \begin{itemize}
        \item Continued researching how to convert our IMU data into the constant pointer data structure that AutoRally wants.
        \item Reviewed the poster put together by Cierra and Tao while I've been caring for my wife and newborn son.
        \item Worked with Tao on finding different simulations to possibly use during Expo.
    \end{itemize}

    \item {\textbf{Problems Encountered}}
    \begin{itemize}
        \item Adjusting to a newborn baby is rough!
        \item Getting accurate state estimation for the vehicle is proving problematic.
        \item We found out that we misunderstood poster submission requirements. We did not submit the printing job to Printing Media Services within the allotted time-frame for Engineering Expo posters.
    \end{itemize}

    \item{\textbf{Plans for next week}}
    \begin{itemize}
        \item Continue to hack away at getting sensor data integrated and experiment with our navigation stack in simulation.
        \item Work on our mid-term submission requirements.
        \begin{itemize}
            \item Written progress report
            \item Progress report video
            \item Expo pitch
            \item Final project write up rough draft.            
        \end{itemize}
        \item Practice project presentation for the advisory board.
    \end{itemize}
\end{itemize}

\subsubsection{Spring Week 6}
\begin{itemize}
    \item {\textbf{Worked On}}
    \begin{itemize}
        \item Worked with Tao testing a simulation package called Stage. It is very light-weight and, used with RVIZ, it allows use to control and environment and display our car much more simply than Gazebo allows. We will likely use this combination, Stage with RVIZ, to present at Expo.
        \item Wrote my section of the mid-term progress report.
        \item Tao, Cierra, and I recorded the mid-term progress video.
        \item Wrote up my section of the write-up rough draft, which was very similar to the progress report.
        \item We worked on our Expo pitch and project presentation. I am doing the intro, Tao is talking about what we did, and Cierra is wrapping things up in the conclusion.
    \end{itemize}

    \item {\textbf{Problems Encountered}}
    \begin{itemize}
        \item We are still struggling with getting the data from the navigation stack to be usable on actual hardware. The main problem is that the simulation is using "perfect" data for the sensors, in real life the sensor give somewhat inaccurate data. This can be countered by adding redundant sensors to increase noise, which smooths out the data, and/or add an odometry encoder to the car to track distance traveled which, added to the data from the sensors, will give the software a more accurate estimation of the state of the car.
    \end{itemize}

    \item{\textbf{Plans for next week}}
    \begin{itemize}
        \item Last minute work on expo presentation.
        \item Build stand for the car.
        \item Give project presentation to the advisory board.
        \item Do Expo.
    \end{itemize}
\end{itemize}

\subsubsection{Spring Week 7}
\begin{itemize}
    \item {\textbf{Worked On}}
    \begin{itemize}
        \item Built stand for the car.
        \begin{itemize}
            \item Designed base and legs to raise and support the vehicle off the ground.
            \item Painted the stand with a gloss finish coat.            
        \end{itemize}
        \item Gave project presentation to the advisory board.
        \item Presented our project at Expo.
    \end{itemize}

    \item {\textbf{Problems Encountered}}
    \begin{itemize}
        \item No major problems encountered this week.
    \end{itemize}

    \item{\textbf{Plans for next week}}
    \begin{itemize}
        \item Continue to try to implement functionality for hardware.
        \item Build out some of the interfaces to allow for future projects to implement the hardware.
        \item Continue working on the instructions/documentation so that future projects will be able to get to the point we are at very quickly and go from there.
        \item Work on the Project Write Up, add more detail and refine the findings and conclusion sections.
    \end{itemize}
\end{itemize}

\subsubsection{Spring Week 8}
\begin{itemize}
    \item \textit{If you were to redo the project from Fall term, what would you tell yourself?}
    \begin{itemize}
        \item Identify critical points of failure. Things that could possibly set us back significantly if they were to fail.
        \item Test those points early, if possible.
        \item Be more proactive about getting help for problems. Don't struggle with a problem for more than a few days before getting help.
    \end{itemize}

    \item \textit{What's the biggest skill you've learned?}
    \begin{itemize}
        \item I learned about the Robotics Operating System (ROS), a general purpose, open source platform for robotics communication and operation. ROS provides a wide range of libraries for communication and control of many kinds of robots, from mechanical arms to rovers, like the car we worked on during Capstone. This project was complicated enough and covered a long enough period of time that I was able to research ROS in depth. I went through beginner tutorials to understand the basics of how ROS works and then applied some of the concepts from the tutorials to our specific needs in our project. I was able to go through a fairly complete learning cycle to become comfortable with how ROS works. In addition, I had no prior experience with autonomous vehicles, this project gave me exposure to the world of autonomy and gave me a greater appreciation and understanding of what is involved for vehicles to operate autonomously.
    \end{itemize}

    \item \textit{What skills do you see yourself using in the future?}
    \begin{itemize}
        \item Computer Science is about solving problems (or trying to...), this project required us to answer the question, "Is high performance autonomous navigation possible on inexpensive hardware?" We had to identify the problem we were trying to solve, research possible techniques/approaches for solving the problem, test our solutions, make adjustments and try again, communicate with a client and conform to requirements, work as a team for a "long" period of time, then arrive at a conclusion and present our findings and conclusion.
        \item These experiences help develop/refine a wide range of skills, both technical and relational, that apply directly to the CS field and life in general.
    \end{itemize}

    \item \textit{What did you like about the project, and what did you not?}
    \begin{itemize}
        \item I liked my client and my team. Cierra and Tao were great teammates and always positive, eager to work on the project. Kevin was a great client, giving us pointers and keeping expectations within reason.
        \item I liked that we had several terms to work on the project, this allowed us to do something fairly significant and complex, versus only having one term as would be the case with a normal course.
        \item I liked the project itself, learning about autonomous vehicles and the current state of the hobbyist-level in this field was very interesting and it was fun to try to get our car working.
        \item I was not a fan of having to do so many "Progress Report" videos. They were quite time-consuming and a little frustrating. Each progress report took around a week to put together, that's about 4 weeks taken from Fall and Winter terms. I realize that accountability needs to be maintained, perhaps the written report would suffice? It seemed like our weekly meetings with our TA were like a running version of the video.
        \item I did not really like the "Wired" article. I get that the point was to practice our expo pitch and interact with others, but grading us on the formatting and style of writing was a bit much.
    \end{itemize}

    \item \textit{What did you learn from your teammates?}
    \begin{itemize}
        \item I learned from them how to have fun with the process. I think at the beginning of the project my demeanor was too serious. Their enthusiasm and general positivism helped me lighten up and have more fun (I don't know if they would agree, haha). I learned more about electronics from Cierra and more about software simulations from Tao.
    \end{itemize}

    \item \textit{If you were the client for this project, would you be satisfied with the work done?}
    \begin{itemize}
        \item I think I would be satisfied. Our objective was to determine if it's possible to make a high performance autonomous RC car with inexpensive hardware and open source software. We put more than sufficient work to come a informed conclusion on the matter.
    \end{itemize}

    \item \textit{If your project were to be continued next year, what do you think needs to be working on?}
    \begin{itemize}
        \item A team would need to take our hardware and software platform and work on creating a state estimator with the sensors and data we are using. The state estimator is need to keep the car in control during autonomous operation.
    \end{itemize}

    \item \textit{Speak a little about your expo experience.}
    \begin{itemize}
        \item Expo was a fun experience for me, overall. We gave a presentation of our project to the industry advisory board and then talked with attendees during the event. It was fun explaining what we did and why we did it. I talked to quite a few kids and it was really fun seeing them interested in the project and answering their questions.
    \end{itemize}
\end{itemize}

\section{Project Poster}

\section{Project Documentation}
\begin{itemize}
    \item How does the project work?  
    \begin{itemize}
        \item What is its structure?  
        \item What is its Theory of Operation?  
        \item Block and flow diagrams are good here.         
    \end{itemize}
    \item How does one install your software, if any?  
    \item How does one run your software?  
    \item Are there any special hardware, OS, or runtime requirements to run your software?  
    \item Any user guides, API documentation, etc?  
\end{itemize}

\section{New Technology Learned}
\begin{itemize}
    \item What web sites were helpful? (Listed in order of helpfulness.)
    \item What, if any, reference books really helped?
    \item Were there any people on campus that were really helpful?
\end{itemize}

\section{What We Learned}
Be honest here -- no B.S.

\subsection{Cierra}
\begin{itemize}
    \item What technical information did you learn?
    \item What non-technical information did you learn?
    \item What have you learned about project work?
    \item What have you learned about project management?
    \item What have you learned about working in teams?
    \item If you could do it all over, what would you do differently?
\end{itemize}

\subsection{Tao}
\begin{itemize}
    \item What technical information did you learn?
    \item What non-technical information did you learn?
    \item What have you learned about project work?
    \item What have you learned about project management?
    \item What have you learned about working in teams?
    \item If you could do it all over, what would you do differently?
\end{itemize}

\subsection{Dan}
\begin{itemize}
    \item What technical information did you learn?
    \item What non-technical information did you learn?
    \item What have you learned about project work?
    \item What have you learned about project management?
    \item What have you learned about working in teams?
    \item If you could do it all over, what would you do differently?
\end{itemize}

\section{Appendix 1}
Essential Code Listings. You don't have to include absolutely everything, but if someone wants to understand your project, there should be enough here to learn from. If you worked within a larger project, something like a patch file might be a good way to go.

\section{Appendix 2}
Anything else you want to include. Photos, etc.

\end{document}

