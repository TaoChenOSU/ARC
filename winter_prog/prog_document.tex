\documentclass[compsoc,draftclsnofoot,onecolumn,10pt]{IEEEtran}
\usepackage[utf8]{inputenc}
\usepackage{color}
\usepackage{url}

\usepackage{enumitem}

\usepackage[letterpaper, margin=.75in]{geometry}
\usepackage{hyperref}
\usepackage{listings}

\usepackage[dvipsnames]{xcolor}
\newcommand*{\SignatureAndDate}[1]{
    \par\noindent\makebox[2.5in]{\hrulefill} \hfill\makebox[2.0in]{\hrulefill}
    \newline\noindent\makebox[2.5in][l]{#1}  \hfill\makebox[2.0in][l]{Date}
}
\usepackage{etoolbox}
\patchcmd{\thebibliography}{\section*}{\subsection}{}{}
\patchcmd{\thebibliography}{\addcontentsline{toc}{section}{\refname}}{}{}{}

\usepackage{graphicx}
\usepackage{float}

\definecolor{dkgreen}{rgb}{0,0.6,0}
\definecolor{gray}{rgb}{0.5,0.5,0.5}
\definecolor{mauve}{rgb}{0.58,0,0.82}

\renewcommand{\lstlistingname}{Code Example} % a listing caption title.

\lstset{
	frame=single,
	language=C,
	columns=flexible,
	numbers=left,
	numbersep=5pt,
	numberstyle=\tiny\color{gray},
	keywordstyle=\color{blue},
	commentstyle=\color{dkgreen},
	stringstyle=\color{mauve},
	breaklines=true,
	breakatwhitespace=true,
	tabsize=4,
	captionpos=b
}

\def\name{Cierra Shawe, Daniel Stoyer, Tao Chen}

%% The following metadata will show up in the PDF properties
\hypersetup{
	colorlinks = false,
	urlcolor = black,
	pdfauthor = {\name},
	pdfkeywords = {ARC,Progress,Report, Winter,2017},
	pdftitle = {Arc Progress Report, Winter 2017},
	pdfsubject = {Progress Report for ARC},
	pdfpagemode = UseNone
}

\def\myversion{1.0 }
\date{}
%
%\usepackage{titlesec}
%
%\usepackage{hyperref}

\parindent = 0.0 in
\parskip = 0.1 in

\setcounter{secnumdepth}{5}

\begin{document}

\begin{titlepage}
\title{
Midterm Progress Report\\
\LARGE
ARC - Autonomous RC\\
Senior Capstone Project\\
Oregon State University\\
Winter 2017
}
\author{Tao Chen, Cierra Shawe, Daniel Stoyer}
\maketitle
\begin{center}
	Version 1.0\\
	\today
\end{center}

\thispagestyle{empty} % gets rid of the "0" page number.
	
\end{titlepage}

\tableofcontents

\newpage

\section{Project purpose and goals} 
% Breifly recaps project purpose and goals
The purpose of the Autonomous RC (ARC) project is to determine if it is possible
to build an autonomous RC vehicle using commodity components, meaning components
that are relatively inexpensive and can be bought at places like Radio
Shack\textsuperscript{\textregistered}, Best Buy\textsuperscript{\textregistered}, or on Amazon. \\
Our goal is to make an RC vehicle navigate autonomous to a given
waypoint/location, preferably at a high rate of speed. Stretch goals are to
make the vehicle drift around corners and parallel park.\\

%\[Be sure to add graphic back in before final render!!\]
\begin{figure}[H]
   	\includegraphics[width=\textwidth]{autorally_platform_header}
    \caption{Drifting example. Image from https://autorally.github.io/}
\end{figure}

While our main goal is to have a functioning autonomous RC vehicle we also hope
that we can produce instructions that RC enthusiasts can follow to produce a
functioning, consumer-grade autonomous RC vehicle of their own.

\section{Cierra}
	\subsection{Current Status}
		% what is the current status of your areas of focus?
		% Be sure to include descriptions of experimental design and use images, screenshots, code blocks
		
	\subsection{Left to do:}
		% What do you have left to do?
		\begin{itemize}
			\item Something to do...
		\end{itemize}
	\subsection{Challenges}
		
		\begin{tabular}{|p{0.3\linewidth}|p{0.3\linewidth}|}
			\hline
			\textbf{Problems} & \textbf{Solutions}\\
			\hline
			Problems that impeded progress. & Specific actions to resolve problems.\\
			\hline
						
		\end{tabular}
		
\section{Tao}
\subsection{Current Status}
My work was mostly on simlution of the platform on the computer. My strategies was to use as much AutoRally's code as possible. As for right now, I still haven't written any code yet. I teared apart the AutoRally package and tried to understand it. I migrated a minimum amount of code to our own package such that the simulation would still run. That being said, when we are ready to build our platform and start implementation, we will use the current package as the foundation. The simulation of the platform will not drive itself. It has to be controlled by a Xbox controller. 

\subsection{Left to do:}
\begin{itemize}
	\item In oder to configure the simulation to accurately simulate the platform, I still edit some parameters in some confiuration and model files. 
	\item The two sensors for vision are not working perfectly. They are outputing the right data. However, I can't visualize their outputs on Rviz.
	\item To make the car drive atonomously with minimum obstacle avoidance. Currently, our package does not have the state estimator, the waypoint follower and the constant speed controller, which are all essential for atonomous driving. I need to integrate those components with sensor data to accomplish this task.
\end{itemize}
\subsection{Challenges}
	Time is the issue. There are only 4 weeks left in this term. It is very hard to set a timeline for this project because we don't have a clear view of what is ahead of us. New issues arise constantly. I will try my best to complete the third item on the "Left to do" list in two weeks. 
\begin{tabular}{|p{0.3\linewidth}|p{0.3\linewidth}|}
	\hline
	\textbf{Problems} & \textbf{Solutions}\\
	\hline
	Problems that impeded progress. & Specific actions to resolve problems.\\
	\hline
	
\end{tabular}

\section{Dan}
\subsection{Current Status}
% what is the current status of your areas of focus?
% Be sure to include descriptions of experimental design and use images, screenshots, code blocks

\subsection{Left to do:}
% What do you have left to do?
\begin{itemize}
	\item Something to do...
\end{itemize}
\subsection{Challenges}

\begin{tabular}{|p{0.3\linewidth}|p{0.3\linewidth}|}
	\hline
	\textbf{Problems} & \textbf{Solutions}\\
	\hline	
	Problems that impeded progress. & Specific actions to resolve problems.\\
	\hline
	
\end{tabular}

\end{document}
