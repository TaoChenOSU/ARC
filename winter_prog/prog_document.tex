\documentclass[compsoc,draftclsnofoot,onecolumn,10pt]{IEEEtran}
\usepackage[utf8]{inputenc}
\usepackage{color}
\usepackage{url}

\usepackage{enumitem}

\usepackage[letterpaper, margin=.75in]{geometry}
\usepackage{hyperref}
\usepackage{listings}

\usepackage[dvipsnames]{xcolor}
\newcommand*{\SignatureAndDate}[1]{
    \par\noindent\makebox[2.5in]{\hrulefill} \hfill\makebox[2.0in]{\hrulefill}
    \newline\noindent\makebox[2.5in][l]{#1}  \hfill\makebox[2.0in][l]{Date}
}
\usepackage{etoolbox}
\patchcmd{\thebibliography}{\section*}{\subsection}{}{}
\patchcmd{\thebibliography}{\addcontentsline{toc}{section}{\refname}}{}{}{}

\usepackage{graphicx}
\usepackage{float}

\definecolor{dkgreen}{rgb}{0,0.6,0}
\definecolor{gray}{rgb}{0.5,0.5,0.5}
\definecolor{mauve}{rgb}{0.58,0,0.82}

\renewcommand{\lstlistingname}{Code Example} % a listing caption title.

\lstset{
	frame=single,
	language=C,
	columns=flexible,
	numbers=left,
	numbersep=5pt,
	numberstyle=\tiny\color{gray},
	keywordstyle=\color{blue},
	commentstyle=\color{dkgreen},
	stringstyle=\color{mauve},
	breaklines=true,
	breakatwhitespace=true,
	tabsize=4,
	captionpos=b
}

\def\name{Cierra Shawe, Daniel Stoyer, Tao Chen}

%% The following metadata will show up in the PDF properties
\hypersetup{
	colorlinks = false,
	urlcolor = black,
	pdfauthor = {\name},
	pdfkeywords = {ARC,Progress,Report, Winter,2017},
	pdftitle = {Arc Progress Report, Winter 2017},
	pdfsubject = {Progress Report for ARC},
	pdfpagemode = UseNone
}

\def\myversion{1.0 }
\date{}
%
%\usepackage{titlesec}
%
%\usepackage{hyperref}

\parindent = 0.0 in
\parskip = 0.1 in

\setcounter{secnumdepth}{5}

\begin{document}

\begin{titlepage}
\title{
Midterm Progress Report\\
\LARGE
ARC - Autonomous RC\\
Senior Capstone Project\\
Oregon State University\\
Winter 2017
}
\author{Tao Chen, Cierra Shawe, Daniel Stoyer}
\maketitle
\begin{center}
	Version 1.0\\
	\today
\end{center}

\thispagestyle{empty} % gets rid of the "0" page number.
	
\end{titlepage}

\tableofcontents

\newpage

\section{Project purpose and goals} 
% Breifly recaps project purpose and goals
The purpose of the Autonomous RC (ARC) project is to determine if it is possible
to build an autonomous RC vehicle using commodity components, meaning components
that are relatively inexpensive and can be bought at places like Radio
Shack\textsuperscript{\textregistered}, Best Buy\textsuperscript{\textregistered}, or on Amazon. \\
Our goal is to make an RC vehicle navigate autonomous to a given
waypoint/location, preferably at a high rate of speed. Stretch goals are to
make the vehicle drift around corners and parallel park.\\

%\[Be sure to add graphic back in before final render!!\]
\begin{figure}[H]
   	\includegraphics[width=\textwidth]{autorally_platform_header}
    \caption{Drifting example. Image from https://autorally.github.io/}
\end{figure}

While our main goal is to have a functioning autonomous RC vehicle we also hope
that we can produce instructions that RC enthusiasts can follow to produce a
functioning, consumer-grade autonomous RC vehicle of their own.

\section{Cierra}
	\subsection{Current Status}
		% what is the current status of your areas of focus?
		% Be sure to include descriptions of experimental design and use images, screenshots, code blocks
		
	\subsection{Left to do:}
		% What do you have left to do?
		\begin{itemize}
			\item Something to do...
		\end{itemize}
	\subsection{Challenges}
		
		\begin{tabular}{|p{0.3\linewidth}|p{0.3\linewidth}|}
			\hline
			\textbf{Problems} & \textbf{Solutions}\\
			\hline
			Problems that impeded progress. & Specific actions to resolve problems.\\
			\hline
						
		\end{tabular}
		
\section{Tao}
\subsection{Current Status}
% what is the current status of your areas of focus?
% Be sure to include descriptions of experimental design and use images, screenshots, code blocks

\subsection{Left to do:}
% What do you have left to do?
\begin{itemize}
	\item Something to do...
\end{itemize}
\subsection{Challenges}

\begin{tabular}{|p{0.3\linewidth}|p{0.3\linewidth}|}
	\hline
	\textbf{Problems} & \textbf{Solutions}\\
	\hline
	Problems that impeded progress. & Specific actions to resolve problems.\\
	\hline
	
\end{tabular}

\section{Dan}
\subsection{Current Status}
% what is the current status of your areas of focus?
% Be sure to include descriptions of experimental design and use images, screenshots, code blocks

\subsection{Left to do:}
% What do you have left to do?
\begin{itemize}
	\item Something to do...
\end{itemize}
\subsection{Challenges}

\begin{tabular}{|p{0.3\linewidth}|p{0.3\linewidth}|}
	\hline
	\textbf{Problems} & \textbf{Solutions}\\
	\hline	
	Problems that impeded progress. & Specific actions to resolve problems.\\
	\hline
	
\end{tabular}

\end{document}
