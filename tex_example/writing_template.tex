\documentclass[draftclsnofoot,onecolumn,10pt]{IEEEtran}
\usepackage[utf8]{inputenc}
\usepackage{color}
\usepackage{url}

\usepackage{enumitem}

\usepackage[letterpaper, margin=.75in]{geometry}

\newcommand{\toc}{\tableofcontents}

\usepackage{hyperref}
\usepackage{listings}

\definecolor{dkgreen}{rgb}{0,0.6,0}
\definecolor{gray}{rgb}{0.5,0.5,0.5}
\definecolor{mauve}{rgb}{0.58,0,0.82}

\renewcommand{\lstlistingname}{Code Example} % a listing caption title.
%\renewcommand{\lstlistlistingname}{List of \lstlistingname s} % list of lists -> list of Thread Program
\lstset{
    frame=single,
    language=C,
    columns=flexible,
    numbers=left,
    numbersep=5pt,
    numberstyle=\tiny\color{gray},
    keywordstyle=\color{blue},
    commentstyle=\color{dkgreen},
    stringstyle=\color{mauve},
    breaklines=true,
    breakatwhitespace=true,
    tabsize=4,
    captionpos=b
}

\def\name{Put your name here}

%% The following metadata will show up in the PDF properties
\hypersetup{
  colorlinks = false,
  urlcolor = black,
  pdfauthor = {\name},
  pdfkeywords = {cs444 "operating systems" files filesystem I/O},
  pdftitle = {CS 444 Weekly Summaries},
  pdfsubject = {CS 444 Weekly Summary 1},
  pdfpagemode = UseNone
}

\parindent = 0.0 in
\parskip = 0.1 in

\begin{document}

\begin{titlepage}
\title{Operating System Feature Comparison: I/O and Provided Functionality\\CS444\\Spring 2016}
\author{Put your name here}
\maketitle
\begin{abstract}
	Abstract goes here.
\end{abstract}

\thispagestyle{empty} % gets rid of the "0" page number.

\end{titlepage}
%\newpage

\tableofcontents

\newpage

\section{Introduction}
A comparison of the I/O devices, schedulers and more of the Windows and FreeBSD operating systems to the Linux I/O implementation.\\
This document details key elements of Windows and FreeBSD I/O implementation and compares each operating system against corresponding Linux I/O implementation.\\
The first section of this document examines Windows I/O implementation and compares it to Linux. The second section examins FreeBSD implementation and compares it to Linux.\\
The conclusion provides an overview of findings from examining Windows and FreeBSD and provides a synthesis of. \cite{windows}

Reference to appendix \ref{appendix:program}
\section{Windows}

\subsection{Devices}

\subsection{I/O Scheduling}

\subsection{Comparison to Linux}

\subsection{PlaceHolder}

\section{FreeBSD}

\subsection{Devices}

\subsection{I/O Scheduling}

\subsection{Comparison to Linux}

\subsection{PlaceHolder}

\section{Conclusion}
My profound findings on the subject matter at hand.

\nocite{linux}
\nocite{bsd}
%\subsection{Windows Processes}
%
%windows cleans up resource in any case.\cite{windows}\\ Viewed at a high-level, a windows\\ 
%
%\subsubsection{Win. Process Structure}
%\begin{itemize}
%    \item Private virtual address space.
%    \item Executable Program
%    \item List of open handles to various system resources that are accessible to all threads in the process.
%    \item Access token: a security context that identifies the user, security groups, privileges and other account states associated with the process.
%    \item Process ID
%    \item At least one thread of execution (although and "empty" process is possible, it is not useful).
%\end{itemize}
%
\newpage
\begin{appendices}

	\section{Example Appendix}
	\label{appendix:program}
	Basic thread program example:
	\begin{lstlisting}[frame=single,caption={Example Caption}]
	int main(int argc, char **argv) {
		pthread_t thr[NUM_THREADS];
		int i, rc;
		/* create a thread_data_t arument array */
		thread_data_t thr_data[NUM_THREADS];

		/* initialize shared data */
		shared_x = 0;

		/* create threads */
		for (i = 0; i < NUM_THREADS; ++i) {
			thr_data[i].tid = i;
			thr_data[i].stuff = (i+1) * NUM_THREADS;
			if ((rc = pthread_create(&thr[i], NULL, thr_func, &thr_data[i]))) {
			fprintf(stderr, "error: pthread_create, rc: %d\n", rc);
			return EXIT_FAILURE;
		}
	}
	/* block until all threads complete */
	for (i = 0; i < NUM_THREADS; ++i) {
		pthread_join(thr[i], NULL);
	}

	return EXIT_SUCCESS;
}
\end{lstlisting}

\end{appendices}

\bibliography{IEEEabrv,mybib}{}
\bibliographystyle{IEEEtran}
\end{document}
