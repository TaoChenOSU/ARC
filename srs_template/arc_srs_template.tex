\documentclass[draftclsnofoot,onecolumn,10pt]{IEEEtran}
\usepackage[utf8]{inputenc}
\usepackage{color}
\usepackage{url}

\usepackage{enumitem}

\usepackage[letterpaper, margin=.75in]{geometry}

\newcommand{\toc}{\tableofcontents}

\usepackage{hyperref}
\usepackage{listings}

\definecolor{dkgreen}{rgb}{0,0.6,0}
\definecolor{gray}{rgb}{0.5,0.5,0.5}
\definecolor{mauve}{rgb}{0.58,0,0.82}

\renewcommand{\lstlistingname}{Code Example} % a listing caption title.

\lstset{
	frame=single,
	language=C,
	columns=flexible,
	numbers=left,
	numbersep=5pt,
	numberstyle=\tiny\color{gray},
	keywordstyle=\color{blue},
	commentstyle=\color{dkgreen},
	stringstyle=\color{mauve},
	breaklines=true,
	breakatwhitespace=true,
	tabsize=4,
	captionpos=b
}

\def\name{Cierra Shawe, Daniel Stoyer, Tao Chen}

%% The following metadata will show up in the PDF properties
\hypersetup{
	colorlinks = false,
	urlcolor = black,
	pdfauthor = {\name},
	pdfkeywords = {$<$SRS requirements, Fall 2016$>$},
	pdftitle = {$<$SRS requirements$>$},
	pdfsubject = {$<$Requirement document for ARCt$>$},
	pdfpagemode = UseNone
}

\def\myversion{1.0 }
\date{}
%
%\usepackage{titlesec}
%
%\usepackage{hyperref}

\parindent = 0.0 in
\parskip = 0.1 in

\begin{document}

\begin{titlepage}
	\title{Software Requirements Specification\\
	ARC - Autonomous RC}
	\author{Tao Chen, Cierra Shawe, Daniel Stoyer}
	\maketitle
	\begin{center}
	Version 0.1\\
	\vspace{1.9cm}
	\today
	\end{center}

	\thispagestyle{empty} % gets rid of the "0" page number.
	
\end{titlepage}

\tableofcontents

\newpage

\section{Introduction} % Cierra, Done
This section provides a scope description and overview of everything included 
in this SRS document. The purpose of this document is described and a list of 
abbreviations and definitions is provided.

\subsection{Purpose} % Cierra, Done
The purpose of this document is to give a detailed description of the requirements 
for the "Autonomous RC System" or ARCS. It will illustrate the purpose and 
complete declaration for the development of system. It will also explain system 
constraints, interface and interactions with other external applications. This document 
is primarily intended to be proposed to a customer for its approval and a reference for 
developing the first version of the system for the development team.

\subsection{Scope} % Cierra, Done
ARCS is a software-hardware interface designed to retrofit RC cars for autonomous 
operation, using commodity hardware. The software and hardware specifications 
should be available free to download and modify at the users will. \par
Users should be able to purchase and install the specified hardware, and implement a 
version that can autonomously navigate to a given destination, or a within a pre-
defined space.  \par
The software will need to be installed on specific hardware, and flashed onto the 
main processing unit, along with control software on a computer or tablet. Hardware 
that we expect to need is a base station that includes a transceiver, an RC retrofitted 
with the main processing unit, a transceiver to send and receive information, a 
controller to send signals to sensors and actuation devices (motors, servos, etc...) and
a vision system to aide in navigation. \par
ARCS will be expected to be able to receive input from a user base station, and react
within the environment based on a destination that the system receives. It should also
be able to navigate to the destination without user intervention, as fast as possible. \par

\subsection{Definitions, Acronyms, and Abbreviations}
This subsection should provide the definitions of all terms, acronyms, and
abbreviations required to properly interpret the SRS. This information may be
provided by reference to one or more appendixes in the SRS or by reference to
other documents.
\begin{enumerate}
	\item \textbf{$<$Term goes here$>$:} $<$Definition goes here$>$
	\item \textbf{$<$Term goes here$>$:} $<$Definition goes here$>$
\end{enumerate}


\subsection{References} % Cierra, Done (for now)
[1] IEEE Software Engineering Standards Committee, ?IEEE Std 830-1998, IEEE 
Recommended Practice for Software Requirements Specifications?, October 20, 1998.

\subsection{Overview} % Cierra, Done 
The remainder of this document includes three sections and the appendixes. \par
Section two provides an overview of system functions and intersystem interaction. 
System constrains and assumptions are also addressed. \par
Section three provides the requirement specification in detailed terms and system 
interface descriptions. Different specification techniques are used in order to specify 
requirements for different audiences. \par
Section four priorities requirements and includes motivation for the chose prioritization 
and discusses why other methods were not chosen. \par
Appendixes at the end of the document include results of the requirement prioritization 
and a release plan based on the requirments.

%%%%%%%%%%%%%%
\section{Overall Description} % Cierra, Done
This section provides a system overview. The system will be explained in its context to 
create a better understanding of how the system interacts with other systems and 
introduce the basic system functionality. This section will also describe what types of 
stakeholders will use the system and what functionality is available for each type. 
Lastly, constraints and assumptions for the system will be presented. 

% This section usually consists of six subsections, as follows:
% \begin{enumerate}
%	\item Product perspective;
%	\item Product functions;
%	\item User characteristics;
%	\item Constraints;
%	\item Assumptions and dependencies;
%	\item Apportioning of requirements.
% \end{enumerate}

%%%%%%%%%%%%%%%%
\subsection{Product Perspective} % Cierra


The ARC system (ARCS) will be designed to integrate into an RC car using 
commodity hardware, and open to anyone who is interested in using it. This 
makes ARCS a component of a larger system, namely the RC car. \par
The basic integration will be as follows:

%%% THIS IS FOR DRAWING THE BLOCK DIAGRAM %%%
\vspace{10cm}
%%%%%%%%%%%%%%%%%%%%%%%%%%%%%%%

This allows for 
% This subsection of the SRS should put the product into perspective with other
% related products. If the product is independent and totally self-contained, it
% should be so stated here. If the SRS defines a product that is a component of a
% larger system, as frequently occurs, then this subsection should relate the
% requirements of that larger system to functionality of the software and should
% identify interfaces between that system and the software.

A block diagram showing the major components of the larger system,
interconnections, and external interfaces can be helpful.

This subsection should also describe how the software operates inside various
constraints. For example, these constraints could include
\begin{enumerate}
	\item System interfaces;
	\item User interfaces;
	\item Hardware interfaces;
	\item Software interfaces;
	\item Communications interfaces;
	\item Memory;
	\item Operations;
	\item Site adaptation requirements.
\end{enumerate}

\subsubsection{System interfaces} %Tao, finished
There are a total of 5 system interfaces where the system can communicate with 
the outside world.
\begin{enumerate}
	\item Sensors: Sensors will have a two-way communication with a secondary 
computer unit, where filtering and smoothing will happen before reliable data will 
be passed to the primary computer. The programs that reside in the secondary unit 
will utilize various methods to generate reliable results based on the raw data. 
At start-up, there will be a script executed by the system to correctly configure 
and calibrate each sensor. Sensors may include: battery sensor, temperature sensor, 
GPS, speed sensor, and IMUs.
	\item Radio: This is the portal of the system where operator/user can monitor 
the status of the vehicle. Different protocols will be implemented for telemetry data, 
which will be displayed to the operator/user. This portal also allows operator/user to 
take control over the computer in case of emergencies. 
	\item Visual unit: This is the interface where the visual unit can pass streams 
of images to the system.
	\item Actuator: The system issues commands to the motors and servos via this 
interface. 
	\item User interface: This interface is a different interface than the radio 
interface even though they both allow humans to interact with the system. The user 
interface will be disable after the vehicle starts maneuvering. This interface allows 
user to input operation modes and desired destinations into the system.


\subsubsection{User interfaces} %Tao, finished
The user interface is a simple, concise GUI. Anyone who knows how to operate a mouse 
will interact with the interface with no problem. A map makes it easier for users to 
pin point destinations and view the current location of the vehicle. An error messages 
will be generated if destination is out of range, meaning that with the onboard battery 
the vehicle won’t be able reach the desire destination. If the direct distance between 
the vehicle and the station exceeds the maximum radio range, an error message will be 
generated as well.\par
With proper training (in less than 30 minutes), one can understand all the indicators 
of the system to know the status of the vehicle.\par
The GUI is a single window/page arrangement. A large portion of the window is dedicated 
to the map. A little section at the bottom is the indicators. Error messages will directly 
appear on the map with alarm to warn the user.\par


\subsubsection{Hardware Interfaces} % Cierra

This should specify the logical characteristics of each interface between the
software product and the hardware components of the system. This includes
configuration characteristics (number of ports, instruction sets, etc.). It also
covers such matters as what devices are to be supported, how they are to be
supported, and protocols. For example, terminal support may specify full-screen
support as opposed to line-by-line support.


\subsubsection{Software Interfaces} % Dan
This should specify the use of other required software products (e.g., a data
management system, an operating system, or a mathematical package), and
interfaces with other application systems (e.g., the linkage between an accounts
receivable system and a general ledger system). For each required software
product, the following should be provided:
\begin{itemize}
	\item Name;
	\item Mnemonic;
	\item Specification number;
	\item Version number;
	\item Source.
\end{itemize}


For each interface, the following should be provided:

\begin{itemize}
	\item Discussion of the purpose of the interfacing software as related to
		this software product.
	\item Definition of the interface in terms of message content and format. It
		is not necessary to detail any well-documented interface, but a
		reference to the document defining the interface is required.
\end{itemize}

\subsubsection{Communications Interfaces} % Dan
This should specify the various interfaces to communications such as local
network protocols, etc.

\subsubsection{Memory constraints}
This should specify any applicable characteristics and limits on primary and
secondary memory.

\subsubsection{Operations}
This should specify the normal and special operations required by the user such as
\begin{enumerate}
	\item The various modes of operations in the user organization (e.g.,
		user-initiated operations);
	\item Periods of interactive operations and periods of unattended operations;
	\item Data processing support functions;
	\item Backup and recovery operations.	content...
\end{enumerate}

This is sometimes specified as part of the User Interfaces section.

\subsubsection{Site Adaptation Requirements}
This should
\begin{enumerate}
	\item Define the requirements for any data or initialization sequences that
		are specific to a given site, mission, or operational mode (e.g., grid
		values, safety limits, etc.);
	\item Specify the site or mission-related features that should be modified
		to adapt the software to a particular installation.a
\end{enumerate}

%%%%%%%%%%%%%%%%%%%%%%%%
\subsection{Product Functions}
This subsection of the SRS should provide a summary of the major functions that
the software will perform. For example, an SRS for an accounting program may use
this part to address customer account maintenance, customer statement, and
invoice preparation without mentioning the vast amount of detail that each of
those functions requires.\par

Sometimes the function summary that is necessary for this part can be taken
directly from the section of the higher-level specification (if one exists) that
allocates particular functions to the software product. Note that for the sake
of clarity
\begin{enumerate}
	\item The functions should be organized in a way that makes the list of
		functions understandable to the customer or to anyone else reading the
		document for the first time.
	\item Textual or graphical methods can be used to show the different
		functions and their relationships. Such a diagram is not intended to
		show a design of a product, but simply shows the logical relationships
		among variables.
\end{enumerate}

\subsection{User Characteristics}
This subsection of the SRS should describe those general characteristics of the
intended users of the product including educational level, experience, and
technical expertise. It should not be used to state specific requirements, but
rather should provide the reasons why certain specific requirements are later
specified in Section 3 of the SRS.

\subsection{Constraints}
This subsection of the SRS should provide a general description of any other
items that will limit the developer's options. These include
\begin{enumerate}
	\item Regulatory policies;
	\item Hardware limitations (e.g., signal timing requirements);
	\item Interfaces to other applications;
	\item Parallel operation;
	\item Audit functions;
	\item Control functions;
	\item Higher-order language requirements;
	\item Signal handshake protocols (e.g., XON-XOFF, ACK-NACK);
	\item Reliability requirements;
	\item Criticality of the application;
	\item Safety and security considerations.
\end{enumerate}

\subsection{Assumptions and Dependencies}
This subsection of the SRS should list each of the factors that affect the
requirements stated in the SRS. These factors are not design constraints on the
software but are, rather, any changes to them that can affect the requirements
in the SRS. For example, an assumption may be that a specific operating system
will be available on the hardware designated for the software product. If, in
fact, the operating system is not available, the SRS would then have to change
accordingly.

\subsection{Apportionment of Requirements}
This subsection of the SRS should identify requirements that may be delayed
until future versions of the system.

\section{Specific Requirements}
This section of the SRS should contain all of the software requirements to a
level of detail sufficient to enable designers to design a system to satisfy
those requirements, and testers to test that the system satisfies those
requirements. Throughout this section, every stated requirement should be
externally perceivable by users, operators, or other external systems. These
requirements should include at a minimum a description of every input (stimulus)
into the system, every output (response) from the system, and all functions
performed by the system in response to an input or in support of an output. As
this is often the largest and most important part of the SRS, the following
principles apply:
\begin{enumerate}
	\item Specific requirements should be stated in conformance with all the
		characteristics described in 4.3.
	\item Specific requirements should be cross-referenced to earlier documents
		that relate.
	\item All requirements should be uniquely identifiable.
	\item Careful attention should be given to organizing the requirements to
		maximize readability.
\end{enumerate}

Before examining specific ways of organizing the requirements it is helpful to
understand the various items that comprise requirements as described in 3.1
through 3.7.

\subsection{External Interfaces}

This should be a detailed description of all inputs into and outputs from the
software system. It should complement the interface descriptions in 5.2 and
should not repeat information there.

It should include both content and format as follows:
\begin{enumerate}
	\item Name of item;
	\item Description of purpose;
	\item Source of input or destination of output;
	\item Valid range, accuracy, and/or tolerance;
	\item Units of measure;
	\item Timing;
	\item Relationships to other inputs/outputs;
	\item Screen formats/organization;
	\item Window formats/organization;
	\item Data formats;
	\item Command formats;
	\item End messages.
\end{enumerate}

\subsection{Functions}
Functional requirements should define the fundamental actions that must take
place in the software in accepting and processing the inputs and in processing
and generating the outputs. These are generally listed as "shall" statements
starting with "The system shall..."

\begin{enumerate}
	\item These include
	\item Validity checks on the inputs
	\item Exact sequence of operations
	\item Responses to abnormal situations, including
		\subitem Overflow
		\subitem Communication facilities
		\subitem Error handling and recovery
	\item Effect of parameters
	\item Relationship of outputs to inputs, including
		\subitem Input/output sequences
		\subitem Formulas for input to output conversion
\end{enumerate}


It may be appropriate to partition the functional requirements into subfunctions
or subprocesses. This does not imply that the software design will also be
partitioned that way.

\subsection{Performance Requirements}
This subsection should specify both the static and the dynamic numerical
requirements placed on the software or on human interaction with the software as
a whole. Static numerical requirements may include the following:

\begin{enumerate}
	\item The number of terminals to be supported;
	\item The number of simultaneous users to be supported;
	\item Amount and type of information to be handled.
\end{enumerate}

Static numerical requirements are sometimes identified under a separate section
entitled Capacity.

Dynamic numerical requirements may include, for example, the numbers of
transactions and tasks and the amount of data to be processed within certain
time periods for both normal and peak workload conditions.

All of these requirements should be stated in measurable terms.

For example,

\textit{95\% of the transactions shall be processed in less than 1 s.}

rather than,

\textit{An operator shall not have to wait for the transaction to complete.}

Numerical limits applied to one specific function are normally specified as part
of the processing subparagraph description of that function.

\subsection{Design Constraints}
This should specify design constraints that can be imposed by other standards,
hardware limitations, etc.

\subsubsection{Standards Compliance}
$<$Included for reference, likely to be removed.$>$
This subsection should specify the requirements derived from existing standards
or regulations. They may include the following:

\begin{enumerate}
	\item Report format;
	\item Data naming;
	\item Accounting procedures;
	\item Audit tracing.
\end{enumerate}

For example, this could specify the requirement for software to trace processing
activity. Such traces are needed for some applications to meet minimum
regulatory or financial standards. An audit trace requirement may, for example,
state that all changes to a payroll database must be recorded in a trace file
with before and after values.

\subsection{Software System Attributes}
There are a number of attributes of software that can serve as requirements. It
is important that required attributes be specified so that their achievement can
be objectively verified. Subclauses 3.6.1 through 3.6.5 provide a partial list
of examples.
$<$We will likely need to come up with more/different attributes$>$

\subsubsection{Reliability}
This should specify the factors required to establish the required reliability
of the software system at time of delivery.

\subsubsection{Organizing the Specific Requirements}
$<$We need to choose our organization which may be a combination of the
following (as opposed to simply one of).$>$\\
For anything but trivial systems the detailed requirements tend to be extensive.
For this reason, it is recommended that careful consideration be given to
organizing these in a manner optimal for understanding. There is no one optimal
organization for all systems. Different classes of systems lend themselves to
different organizations of requirements in Section 3 of the SRS. Some of these
organizations are described in 5.3.7.1 through 5.3.7.7


\section{Other Nonfunctional Requirements}

\subsection{Performance Requirements}
$<$If there are performance requirements for the product under various 
circumstances, state them here and explain their rationale, to help the 
developers understand the intent and make suitable design choices. Specify the 
timing relationships for real time systems. Make such requirements as specific 
as possible. You may need to state performance requirements for individual 
functional requirements or features.$>$

\subsection{Safety Requirements}
$<$Specify those requirements that are concerned with possible loss, damage, or 
harm that could result from the use of the product. Define any safeguards or 
actions that must be taken, as well as actions that must be prevented. Refer to 
any external policies or regulations that state safety issues that affect the 
product’s design or use. Define any safety certifications that must be 
satisfied.$>$

\section{Other Requirements}
$<$Define any other requirements not covered elsewhere in the SRS. This might 
include database requirements, internationalization requirements, legal 
requirements, reuse objectives for the project, and so on. Add any new sections 
that are pertinent to the project.$>$

\section{Index}

\section{Appendix}

\subsection{Appendix A: Glossary}
%see https://en.wikibooks.org/wiki/LaTeX/Glossary
% $<$Define all the terms necessary to properly interpret the SRS, including 
% acronyms and abbreviations. You may wish to build a separate glossary that spans 
% multiple projects or the entire organization, and just include terms specific to 
% a single project in each SRS.$>$

\subsection{Appendix B: Analysis Models}
% $<$Optionally, include any pertinent analysis models, such as data flow 
% diagrams, class diagrams, state-transition diagrams, or entity-relationship 
% diagrams.$>$

\subsection{Appendix C: To Be Determined List}
$<$Collect a numbered list of the TBD (to be determined) references that remain 
in the SRS so they can be tracked to closure.$>$

\end{document}
