%Copyright 2014 Jean-Philippe Eisenbarth
%This program is free software: you can 
%redistribute it and/or modify it under the terms of the GNU General Public 
%License as published by the Free Software Foundation, either version 3 of the 
%License, or (at your option) any later version.
%This program is distributed in the hope that it will be useful,but WITHOUT ANY 
%WARRANTY; without even the implied warranty of MERCHANTABILITY or FITNESS FOR A 
%PARTICULAR PURPOSE. See the GNU General Public License for more details.
%You should have received a copy of the GNU General Public License along with 
%this program.  If not, see <http://www.gnu.org/licenses/>.

%Based on the code of Yiannis Lazarides
%http://tex.stackexchange.com/questions/42602/software-requirements-specificatio
%n-with-latex
%http://tex.stackexchange.com/users/963/yiannis-lazarides
%Also based on the template of Karl E. Wiegers
%http://www.se.rit.edu/~emad/teaching/slides/srs_template_sep14.pdf
%http://karlwiegers.com
\documentclass{scrreprt}
\usepackage{enumitem}
\usepackage{listings}
\usepackage{underscore}
\usepackage[bookmarks=true]{hyperref}
\usepackage[utf8]{inputenc}
\usepackage[english]{babel}
\hypersetup{
	bookmarks=false,    % show bookmarks bar?
	pdftitle={Software Requirement Specification},    % title
	pdfauthor={Dan Stoyer},                     % author
	pdfsubject={Autonomous RC (ARC)},           % subject of the document
	pdfkeywords={autonomous, RC, ARC, self-guided, car, neato}, % list of keywords
	colorlinks=true,       % false: boxed links; true: colored links
	linkcolor=blue,       % color of internal links
	citecolor=black,       % color of links to bibliography
	filecolor=black,        % color of file links
	urlcolor=purple,        % color of external links
	linktoc=page            % only page is linked
}%
\def\myversion{1.0 }
\date{}
%\title
\usepackage{titlesec}

\usepackage{hyperref}

\parindent = 0.0 in
\parskip = 0.1 in

\begin{document}

\begin{flushleft}
	\rule{16cm}{5pt}\vskip1cm
	\begin{bfseries}
		\Huge{Software Requirements\\ Specification}\\
		\vspace{1.9cm}
		for\\
		\vspace{1.9cm}
		Autonomous RC (ARC)\\
		\vspace{1.9cm}
		\LARGE{Version \myversion approved}\\
		\vspace{1.9cm}
		Prepared by $<$author$>$\\
		\vspace{1.9cm}
		$<$Organization$>$\\
		\vspace{1.9cm}
		\today\\
	\end{bfseries}
\end{flushleft}

\tableofcontents

\chapter{Introduction}

The introduction of the SRS should provide an overview of the entire SRS. It
should contain the following subsections:
\begin{enumerate}
	\item Purpose;
	\item Scope;
	\item Definitions, acronyms, and abbreviations;
	\item References;
	\item Overview.
\end{enumerate}


\section{Purpose}
This subsection should
\begin{enumerate}
	\item Delineate the purpose of the SRS;
	\item Specify the intended audience for the SRS.
\end{enumerate}

\section{Scope}
This subsection should
\begin{enumerate}
	\item Identify the software product(s) to be produced by name (e.g., Host
		DBMS, Report Generator, etc.);
	\item Explain what the software product(s) will and, if necessary, will not
		do;
	\item Describe the application of th software being specified, including
		relevant benefits, objectives, and goals;
	\item Be consistent with similar statements in higher-level specifications
		(e.g., the system requirements specification), if they exist.
\end{enumerate}

\section{Definitions, Acronyms, and Abbreviations}
This subsection should provide the definitions of all terms, acronyms, and
abbreviations required to properly interpret the SRS. This information may be
provided by reference to one or more appendixes in the SRS or by reference to
other documents.
\begin{enumerate}
	\item \textbf{$<$Term goes here$>$:} $<$Definition goes here$>$
	\item \textbf{$<$Term goes here$>$:} $<$Definition goes here$>$
\end{enumerate}


\section{References}
This subsection should
\begin{enumerate}
	\item Provide a complete list of all documents referenced elsewhere in the SRS;
	\item Identify each document by title, report number (if applicable), date,
		and publishing organization;
	\item Specify the sources from which the references can be obtained.
	\item This information may be provided by reference to an appendix or to
		another document.
\end{enumerate}

\section{Overview}
This subsection should
\begin{enumerate}
	\item Describe what the rest of the SRS contains;
	\item Explain how the SRS is organized.
\end{enumerate}

\chapter{Overall description}
This section of the SRS should describe the general factors that affect the
product and its requirements. This section does not state specific requirements.
Instead, it provides a background for those requirements, which are defined in
detail in Section 3 of the SRS, and makes them easier to understand.\\
This section usually consists of six subsections, as follows:
\begin{enumerate}
	\item Product perspective;
	\item Product functions;
	\item User characteristics;
	\item Constraints;
	\item Assumptions and dependencies;
	\item Apportioning of requirements.
\end{enumerate}

\section{Product perspective}
This subsection of the SRS should put the product into perspective with other
related products. If the product is independent and totally self-contained, it
should be so stated here. If the SRS defines a product that is a component of a
larger system, as frequently occurs, then this subsection should relate the
requirements of that larger system to functionality of the software and should
identify interfaces between that system and the software.

A block diagram showing the major components of the larger system,
interconnections, and external interfaces can be helpful.

This subsection should also describe how the software operates inside various
constraints. For example, these constraints could include
\begin{enumerate}
	\item System interfaces;
	\item User interfaces;
	\item Hardware interfaces;
	\item Software interfaces;
	\item Communications interfaces;
	\item Memory;
	\item Operations;
	\item Site adaptation requirements.
\end{enumerate}

\subsection{System interfaces}
This should list each system interface and identify the functionality of the
software to accomplish the system requirement and the interface description to
match the system.

\subsection{User interfaces}
This should specify the following:

1. The logical characteristics of each interface between the software product
and its users. This includes those configuration characteristics (e.g., required
screen formats, page or window layouts, content of any reports or menus, or
availability of programmable function keys) necessary to accomplish the software
requirements.\par

2. All the aspects of optimizing the interface with the person who must use the
system. This may simply comprise a list of do's and don'ts on how the system
will appear to the user. One example may be a requirement for the option of long
or short error messages. Like all others, these requirements should be
verifiable, e.g., “a clerk typist grade 4 can do function  in min after 1 h of
training” rather than “a typist can do function .” (This may also be
specified in the Software System Attributes under a section titled Ease of
Use.)

\subsection{Hardware intefaces}
This should specify the logical characteristics of each interface between the
software product and the hardware components of the system. This includes
configuration characteristics (number of ports, instruction sets, etc.). It also
covers such matters as what devices are to be supported, how they are to be
supported, and protocols. For example, terminal support may specify full-screen
support as opposed to line-by-line support.

\subsection{Software interfaces}
This should specify the use of other required software products (e.g., a data
management system, an operating system, or a mathematical package), and
interfaces with other application systems (e.g., the linkage between an accounts
receivable system and a general ledger system). For each required software
product, the following should be provided:
\begin{itemize}
	\item Name;
	\item Mnemonic;
	\item Specification number;
	\item Version number;
	\item Source.
\end{itemize}


For each interface, the following should be provided:
\begin{itemize}
	\item Discussion of the purpose of the interfacing software as related to
		this software product.
	\item Definition of the interface in terms of message content and format. It
		is not necessary to detail any well-documented interface, but a
		reference to the document defining the interface is required.
\end{itemize}

\subsection{Communications interfaces}
This should specify the various interfaces to communications such as local
network protocols, etc.

\subsection{Memory constraints}
This should specify any applicable characteristics and limits on primary and
secondary memory.

\subsection{Operations}
This should specify the normal and special operations required by the user such as
\begin{enumerate}
	\item The various modes of operations in the user organization (e.g.,
		user-initiated operations);
	\item Periods of interactive operations and periods of unattended operations;
	\item Data processing support functions;
	\item Backup and recovery operations.	content...
\end{enumerate}

This is sometimes specified as part of the User Interfaces section.

\subsection{Site adaptation requirements}
This should
\begin{enumerate}
	\item Define the requirements for any data or initialization sequences that
		are specific to a given site, mission, or operational mode (e.g., grid
		values, safety limits, etc.);
	\item Specify the site or mission-related features that should be modified
		to adapt the software to a particular installation.a
\end{enumerate}

\section{Product functions}
This subsection of the SRS should provide a summary of the major functions that
the software will perform. For example, an SRS for an accounting program may use
this part to address customer account maintenance, customer statement, and
invoice preparation without mentioning the vast amount of detail that each of
those functions requires.\par

Sometimes the function summary that is necessary for this part can be taken
directly from the section of the higher-level specification (if one exists) that
allocates particular functions to the software product. Note that for the sake
of clarity
\begin{enumerate}
	\item The functions should be organized in a way that makes the list of
		functions understandable to the customer or to anyone else reading the
		document for the first time.
	\item Textual or graphical methods can be used to show the different
		functions and their relationships. Such a diagram is not intended to
		show a design of a product, but simply shows the logical relationships
		among variables.
\end{enumerate}

\section{User characteristics}
This subsection of the SRS should describe those general characteristics of the
intended users of the product including educational level, experience, and
technical expertise. It should not be used to state specific requirements, but
rather should provide the reasons why certain specific requirements are later
specified in Section 3 of the SRS.

\section{Constraints}
This subsection of the SRS should provide a general description of any other
items that will limit the developer's options. These include
\begin{enumerate}
	\item Regulatory policies;
	\item Hardware limitations (e.g., signal timing requirements);
	\item Interfaces to other applications;
	\item Parallel operation;
	\item Audit functions;
	\item Control functions;
	\item Higher-order language requirements;
	\item Signal handshake protocols (e.g., XON-XOFF, ACK-NACK);
	\item Reliability requirements;
	\item Criticality of the application;
	\item Safety and security considerations.
\end{enumerate}

\section{Assumptions and dependencies}
This subsection of the SRS should list each of the factors that affect the
requirements stated in the SRS. These factors are not design constraints on the
software but are, rather, any changes to them that can affect the requirements
in the SRS. For example, an assumption may be that a specific operating system
will be available on the hardware designated for the software product. If, in
fact, the operating system is not available, the SRS would then have to change
accordingly.

\section{Apportionment of requirements}
This subsection of the SRS should identify requirements that may be delayed
until future versions of the system.

\chapter{Specific requirements}
This section of the SRS should contain all of the software requirements to a
level of detail sufficient to enable designers to design a system to satisfy
those requirements, and testers to test that the system satisfies those
requirements. Throughout this section, every stated requirement should be
externally perceivable by users, operators, or other external systems. These
requirements should include at a minimum a description of every input (stimulus)
into the system, every output (response) from the system, and all functions
performed by the system in response to an input or in support of an output. As
this is often the largest and most important part of the SRS, the following
principles apply:
\begin{enumerate}
	\item Specific requirements should be stated in conformance with all the
		characteristics described in 4.3.
	\item Specific requirements should be cross-referenced to earlier documents
		that relate.
	\item All requirements should be uniquely identifiable.
	\item Careful attention should be given to organizing the requirements to
		maximize readability.
\end{enumerate}

Before examining specific ways of organizing the requirements it is helpful to
understand the various items that comprise requirements as described in 3.1
through 3.7.

\section{External interfaces}

This should be a detailed description of all inputs into and outputs from the
software system. It should complement the interface descriptions in 5.2 and
should not repeat information there.

It should include both content and format as follows:
\begin{enumerate}
	\item Name of item;
	\item Description of purpose;
	\item Source of input or destination of output;
	\item Valid range, accuracy, and/or tolerance;
	\item Units of measure;
	\item Timing;
	\item Relationships to other inputs/outputs;
	\item Screen formats/organization;
	\item Window formats/organization;
	\item Data formats;
	\item Command formats;
	\item End messages.
\end{enumerate}

\section{Functions}
Functional requirements should define the fundamental actions that must take
place in the software in accepting and processing the inputs and in processing
and generating the outputs. These are generally listed as "shall" statements
starting with "The system shall..."

\begin{enumerate}
	\item These include
	\item Validity checks on the inputs
	\item Exact sequence of operations
	\item Responses to abnormal situations, including
		\subitem Overflow
		\subitem Communication facilities
		\subitem Error handling and recovery
	\item Effect of parameters
	\item Relationship of outputs to inputs, including
		\subitem Input/output sequences
		\subitem Formulas for input to output conversion
\end{enumerate}


It may be appropriate to partition the functional requirements into subfunctions
or subprocesses. This does not imply that the software design will also be
partitioned that way.

\section{Performance requirements}
This subsection should specify both the static and the dynamic numerical
requirements placed on the software or on human interaction with the software as
a whole. Static numerical requirements may include the following:

\begin{enumerate}
	\item The number of terminals to be supported;
	\item The number of simultaneous users to be supported;
	\item Amount and type of information to be handled.
\end{enumerate}

Static numerical requirements are sometimes identified under a separate section
entitled Capacity.

Dynamic numerical requirements may include, for example, the numbers of
transactions and tasks and the amount of data to be processed within certain
time periods for both normal and peak workload conditions.

All of these requirements should be stated in measurable terms.

For example,

\textit{95\% of the transactions shall be processed in less than 1 s.}

rather than,

\textit{An operator shall not have to wait for the transaction to complete.}

Numerical limits applied to one specific function are normally specified as part
of the processing subparagraph description of that function.

\section{Logical database requirements}
$<$Included for reference, likely to be removed.$>$
This should specify the logical requirements for any information that is to be
placed into a database. This may include the following:

\begin{enumerate}
	\item Types of information used by various functions;
	\item Frequency of use;
	\item Accessing capabilities;
	\item Data entities and their relationships;
	\item Integrity constraints;
	\item Data retention requirements.
\end{enumerate}

\section{Design constraints}
This should specify design constraints that can be imposed by other standards,
hardware limitations, etc.


\subsection{Standards compliance}
$<$Included for reference, likely to be removed.$>$
This subsection should specify the requirements derived from existing standards
or regulations. They may include the following:

\begin{enumerate}
	\item Report format;
	\item Data naming;
	\item Accounting procedures;
	\item Audit tracing.
\end{enumerate}

For example, this could specify the requirement for software to trace processing
activity. Such traces are needed for some applications to meet minimum
regulatory or financial standards. An audit trace requirement may, for example,
state that all changes to a payroll database must be recorded in a trace file
with before and after values.

\section{Software system attributes}
There are a number of attributes of software that can serve as requirements. It
is important that required attributes be specified so that their achievement can
be objectively verified. Subclauses 3.6.1 through 3.6.5 provide a partial list
of examples.
$<$We will likely need to come up with more/different attributes$>$

\subsection{Reliability}
This should specify the factors required to establish the required reliability
of the software system at time of delivery.

\subsection{Availability}
This should specify the factors required to guarantee a defined availability
level for the entire system such as checkpoint, recovery, and restart.

\subsection{Security}
$<$Included for reference, likely to be removed.$>$
This should specify the factors that protect the software from accidental or
malicious access, use, modification, destruction, or disclosure. Specific
requirements in this area could include the need to

\begin{enumerate}
	\item Utilize certain cryptographical techniques;
	\item Keep specific log or history data sets;
	\item Assign certain functions to different modules;
	\item Restrict communications between some areas of the program;
	\item Check data integrity for critical variables.
\end{enumerate}

\subsection{Maintainability}
This should specify attributes of software that relate to the ease of
maintenance of the software itself. There may be some requirement for certain
modularity, interfaces, complexity, etc. Requirements should not be placed here
just because they are thought to be good design practices.

\subsection{Portability}
This should specify attributes of software that relate to the ease of porting
the software to other host machines and/or operating systems. This may include
the following:

\begin{enumerate}
	\item Percentage of components with host-dependent code;
	\item Percentage of code that is host dependent;
	\item Use of a proven portable language;
	\item Use of a particular compiler or language subset;
	\item Use of a particular operating system.
\end{enumerate}

\subsection{Organizing the specific requirements}
$<$We need to choose our organization which may be a combination of the
following (as opposed to simply one of).$>$\\
For anything but trivial systems the detailed requirements tend to be extensive.
For this reason, it is recommended that careful consideration be given to
organizing these in a manner optimal for understanding. There is no one optimal
organization for all systems. Different classes of systems lend themselves to
different organizations of requirements in Section 3 of the SRS. Some of these
organizations are described in 5.3.7.1 through 5.3.7.7


\chapter{Other Nonfunctional Requirements}

\section{Performance Requirements}
$<$If there are performance requirements for the product under various 
circumstances, state them here and explain their rationale, to help the 
developers understand the intent and make suitable design choices. Specify the 
timing relationships for real time systems. Make such requirements as specific 
as possible. You may need to state performance requirements for individual 
functional requirements or features.$>$
\section{Safety Requirements}
$<$Specify those requirements that are concerned with possible loss, damage, or 
harm that could result from the use of the product. Define any safeguards or 
actions that must be taken, as well as actions that must be prevented. Refer to 
any external policies or regulations that state safety issues that affect the 
product’s design or use. Define any safety certifications that must be 
satisfied.$>$

\chapter{Other Requirements}
$<$Define any other requirements not covered elsewhere in the SRS. This might 
include database requirements, internationalization requirements, legal 
requirements, reuse objectives for the project, and so on. Add any new sections 
that are pertinent to the project.$>$

\chapter{Index}

\chapter{Appendix}

\section{Appendix A: Glossary}
%see https://en.wikibooks.org/wiki/LaTeX/Glossary
$<$Define all the terms necessary to properly interpret the SRS, including 
acronyms and abbreviations. You may wish to build a separate glossary that spans 
multiple projects or the entire organization, and just include terms specific to 
a single project in each SRS.$>$

\section{Appendix B: Analysis Models}
$<$Optionally, include any pertinent analysis models, such as data flow 
diagrams, class diagrams, state-transition diagrams, or entity-relationship 
diagrams.$>$

\section{Appendix C: To Be Determined List}
$<$Collect a numbered list of the TBD (to be determined) references that remain 
in the SRS so they can be tracked to closure.$>$

\end{document}
