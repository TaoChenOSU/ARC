\documentclass[compsoc,draftclsnofoot,onecolumn,10pt]{IEEEtran}
\usepackage[utf8]{inputenc}
\usepackage{color}
\usepackage{url}

\usepackage{enumitem}

\usepackage[letterpaper, margin=.75in]{geometry}

\newcommand{\toc}{\tableofcontents}
\newcommand*{\SignatureAndDate}[1]{
    \par\noindent\makebox[2.5in]{\hrulefill} \hfill\makebox[2.0in]{\hrulefill}
    \newline\noindent\makebox[2.5in][l]{#1}  \hfill\makebox[2.0in][l]{Date}
}

\usepackage{hyperref}
\usepackage{listings}

\definecolor{dkgreen}{rgb}{0,0.6,0}
\definecolor{gray}{rgb}{0.5,0.5,0.5}
\definecolor{mauve}{rgb}{0.58,0,0.82}

\renewcommand{\lstlistingname}{Code Example} % a listing caption title.
%\renewcommand{\lstlistlistingname}{List of \lstlistingname s} % list of lists -> list of Thread Program
\lstset{
    frame=single,
    language=C,
    columns=flexible,
    numbers=left,
    numbersep=5pt,
    numberstyle=\tiny\color{gray},
    keywordstyle=\color{blue},
    commentstyle=\color{dkgreen},
    stringstyle=\color{mauve},
    breaklines=true,
    breakatwhitespace=true,
    tabsize=4,
    captionpos=b
}

\def\name{Put your name here}

%% The following metadata will show up in the PDF properties
\hypersetup{
  colorlinks = false,
  urlcolor = black,
  pdfauthor = {\name},
  pdfkeywords = {Capstone Tech Review},
  pdftitle = {Capstone Tech Review},
  pdfsubject = {Tech Review},
  pdfpagemode = UseNone
}

\setlength{\parindent}{0.0 in}
\setlength{\parskip}{0.2 in}

\begin{document}

\begin{titlepage}
\title{
Group 44 - ARC \\
Tech Review\\
\LARGE
Senior Capstone Project\\
Oregon State University\\
Fall 2016
}

\author{Tao Chen, Cierra Shawe, Daniel Stoyer}
\maketitle

\begin{abstract}

\end{abstract}

\thispagestyle{empty} % gets rid of the "0" page number.

\end{titlepage}
\newpage

\tableofcontents

\newpage

\section{Vision System Options - Cierra} % Done
For autonomous operation, vision systems are critical. 
The three main options include stereoscopic cameras, Infrared (IR) based systems such as Microsoft's Kinect, and Light Detection And Ranging (LiDAR) vision systems. 
With the exception of some forms of LiDar, all of these methods require what is called a disparity map, which creates a 3D image of the surface, that can be used for telling which objects in an image are closest or farthest away. 

\subsection{Stereo Vision}
Stereo-vision uses two different cameras to create disparity maps in order to create a sense of depth. 
This is similar to how our eyes work. 
The biggest benefit to a stereoscopic camera system is the ability to detect objects outdoors, as the cameras are able to function with vast amounts of ultraviolet (UV) light. 
IR LEDs can also be used in order to illuminate an area at night, also allowing for night-time navigation. 
Another pro to stereoscopic vision, is the cost is relatively low, as cameras can be obtained for under \$10 a piece.  
One of the challenges of stereo vision is the computational power required. 
Another is clarity of the disparity map without post processing of images, which makes real-time operation more difficult. 
\cite{acs}
OpenCV \cite{opencv} contains many examples of how to configure and process stereoscopic images, and is one of the largest vision resources. It can be used to synchronize and create the the disparity map, which can then be used by another part of our system for decision making. 

\subsection{IR Camera's such as Kinect and RealSense}
Using an infrared point map, these cameras are able to tell the disparity between the points, which helps in creating disparity maps. The most popular example of an IR camera system, is the Microsoft Kinect. 
A big advantage to IR camera's, is the ability to function in low and non-natural lighting conditions, due to using the infrared spectrum, rather than only using the visible spectrum. 
The biggest problem with IR cameras, the functionality is greatly reduced outdoors, due to massive amount of infrared waves from the sun. 
IR cameras don't meet our requirement of being able to use the vision system reliably outdoors.

\subsection{Lidar}
LiDAR works by using laser pulses to detect range. 
By detecting different pulse signatures, it is able to take very precise distance measurements. 
LiDAR is a great way to form point clouds, however, the cost makes the product unreasonable for most people. 
A basic SICK LiDAR unite costs around \$2,000 USD for a unit with 10m accuracy. cite{sick} https://www.sick.com/us/en/product-portfolio/detection-and-ranging-solutions/2d-laser-scanners/tim3xx/c/g205751\\
A 2D RPLiDAR module is a lower cost alternative, at \$449, however, the manufacture says that it will not preform well in direct sunlight, due to using IR lasers. http://www.robotshop.com/en/rplidar-a2-360-laser-scanner.html
The RPLiDAR also only has accurate measurements up to 6 meters.\\
A third option for LiDAR, has yet to come to market. 
The Sweep LiDAR unit by Scanse will be released in January of 2017. 
The Sweep unit claims to be a LiDAR unit available for all.
Costing \$349, the unit has the ability to scan 360 degrees, create points up to 40m away, and is able to function in "noisy" environments, such as outdoors. \\
The Sweep's distance capabilities and ability to be used almost any lighting condition, including outdoors, would make this a great candidate for our project.
The caveat to the Sweep, is it will not be released until January. \par
The LiDAR systems that would be available to our group, would not meet the requirement of being able to function consistently in an outdoor environment, or do not fall under the category of commodity hardware. 

\subsection{Our choice}
Due to the need to be able to navigate in outdoor environments, our team will start out attempting to use stereoscopic imaging as our primary vision system. We will do this using the OpenCV library to analyze the images, and create the disparity map that can be used for other purposes. If we have the computational power to post-process disparity maps in real time, we will attempt to do so. \\
If time allows, given we are able to obtain a unit, our team could investigate the use of the Sweep LiDAR system for navigation in place of stereoscopic imaging. 

\section{Sensors - Cierra}
\subsection{GPS - External}
\subsection{GPS - PXFmini}
\subsection{Internal Measurement Unit (IMU) - External}
\subsection{IMU - PXFmini}


\section{System Control and Synchronization - Cierra}
IDK WHAT TO PUT IN THIS SYSTEM
\subsection{Intel NUC}

\subsection{Intel UPboard}

\subsection{Raspberry Pi 3}


\section{Image Analysis Software - Dan}
Image analysis, for the ARC project, is the processing of visual data received
from cameras into deterministic information, such as pathfinding, or spacial
awareness. This is the primary means for our autonomous vehicle to assess its
surroundings and find its way to a given waypoint while avoiding obstacles. We
require software that is freely available for use (via fairly liberal open
source licensing), known to be correct (works well) with little modification
needed, and has relatively easy to use API libraries.

\subsection{DroneKit-Python}
DroneKit-Python (http://python.dronekit.io/) is part of the DroneKit ecosystem (dronekit.io)\par
more to follow...

\subsection{ArduPilot}
ArduPilot (http://ardupilot.org/rover/index.html)\par
more to follow...

\subsection{Image analysis choice}

\section{Telemetry Radio Communication - Dan}
Telemetry is the transmition of measurement data (velocity, angle, rotation, etc.)
by radio to some other place. This data is important for the human user to
know the current state of the vehicle. This is especially important for
autonomous operation, as the vehicle may not be operating within line of
sight. Telemetry transmission is well-established, so we will not be comparing
vastly different transmission technologies, such as long range (MHz radio frequencies)
versus short-range (bluetooth) where the advantages of ranges of 2-15+
kilometers obviously outweigh ranges of 20-100 meters.

The main criteria for consideration are:
\begin{itemize}
	\item Cost
		\subitem One of our main goals with ARC is to keep the costs low.
	\item Power consumption
		\subitem We have limited power available, therefore we need power consumption to
	be low.
	\item Ease of use
		\subitem The radio needs to be easily integrated into the autopilot system. This
	means it needs to have a developed API with little no modification
	required.
	\item Form factor
		\subitem The size and weight needs to be small and light. If it is too bulky, we
	might not have space on the vehicle. If it is too heavy, more power will
	be required to operate the drive system and will drain the battery
	faster.
\end{itemize}


\subsection{3DR 915 MHz Transceiver}  ((https://store.3dr.com/products/915-mhz-telemetry-radio)\par

The 3DR 915 MHz telemetry radio has a cost of \$39.99 USD for two radios. It is
powered by the autopilot telemetry port (+5v) which means that has low power
consumption. This radio transceiver has open source firmware, a robust API,
and is fully compatible with PX4 Pro, DroneKit, and ArduPilot, using the
MAVLink protocol, which will allow us to implement telemetry transmission with
little to no modification of the API, should we use one of those autopilot
systems.
The form factor has dimensions of 25.5 x 53 x 11 mm (including case, but not
antenna) at 11.5 grams (without antenna).
The range of this transceiver is from 300 meters to several kilometers,
depending on the antenna arrangement.

Biggest advantages: inexpensive, small form factor, low power consumption.

Biggest disavantages: range out of the box could be as low as 300 meters.


\subsection{RFD900 Radio Modem}
(http://ardupilot.org/copter/docs/common-rfd900.html)(http://store.jdrones.com/jD\_RD900Plus\_Telemetry\_Bundle\_p/rf900set02.htm)\par

The RFD900 Radio Modem has a cost of \$259.99 USD for two radios. It requires
seperate +5v power for operation which means that it has high power
consumption. This radio has open source firmware, a robust API, and is fully
compatible with PX4 Pro, DroneKit, and ArduPilot, using the MAVLink protocol,
which will allow us to implement telemetry transmission with little to no
modification of the API, should we use one of those autopilot systems.

The form factor has dimensions of 70 x 40 x 23mm (including case, but not
antenna) at 14.5 grams (without
antenna).
The range of this transceiver is 25+ kilometers.

Biggest advantages: ultra long range.

Biggest disadvantages: expensive, large size.

\subsection{Openpilot OPLink Mini Ground and Air Station 433 MHz}
(https://hobbyking.com/en\_us/openpilot-oplink-mini-ground-station-433-mhz.html)
(http://www.banggood.com/Openpilot-OPLINK-Mini-Radio-Telemetry-AIR-And-Ground-For-MINI-CC3D-Revolution-p-1018904.html)\par

The OPLink Mini Ground Station has a cost of \$26.59 USD for two radios. It
requires input voltage of +5v and can be powered off the autopilot telemetry
port which means that it has low power consumption. This radio has open source
firmware but is only compatible with the LibrePilot RC control system. 
The form factor has dimensions of 38 x 23 x mm (including case, but not
antenna) at 4 grams (without antenna).
The range of this radio is not known, but based on the power requirements and
frequency it likely has less range than the 3DR 915 MHz radio.

Biggest advantages: smallest form factor (only 4 grams), lowest cost (\$26.59
USD)

Biggest disavantages: Only works with the LibrePilot control system.


\subsection{Telemetry radio choice}
The 3DR 915 MHz Transceiver is our selection for the telemetry radio.
While the OPLink Mini Ground Station was significantly smaller, lighter, and
cheaper than the other two, its implementation being tied solely to LibrePilot
was a deal breaker (more information on LibrePilot can be found in the User
Interface evaluation).
The RF900 Radio Modem would have been a good choice, it has fantastic range
and all the API options we were looking for. But it had a significantly larger
form factor, required a separate power supply, and was quite expensive at
\$259.99. Put together, these facts eliminated the RF900 as a viable option.
The 3DR 915 MHz Transceiver is a good balance of cost, performance, and size. The
cost of \$39.99 for two radios, the ability to power the autopilot off the
telemetry port, and the portability of its APIs and their ease of use, puts the
3DR 915 MHz at the top of our list and the clear choice for the telemetry
radio going forward.

\section{User Interface - Dan}

\subsection{QGroundControl}
(http://qgroundcontrol.com/downloads/)
more to follow...

\subsection{DroneKit-Android}
(http://android.dronekit.io/)
more to follow...

\subsection{LibrePilot}
(https://www.librepilot.org/site/index.html)
more to follow...

\subsection{User interface choice}
QgroundControl, for many very good reasons (no, this isn't the actual reasoning).
more to follow...

\end{document}
