\documentclass[compsoc,draftclsnofoot,onecolumn,10pt]{IEEEtran}
\usepackage[utf8]{inputenc}
\usepackage{color}
\usepackage{url}

\usepackage{enumitem}

\usepackage[letterpaper, margin=.75in]{geometry}

\newcommand{\toc}{\tableofcontents}
\newcommand*{\SignatureAndDate}[1]{
    \par\noindent\makebox[2.5in]{\hrulefill} \hfill\makebox[2.0in]{\hrulefill}
    \newline\noindent\makebox[2.5in][l]{#1}  \hfill\makebox[2.0in][l]{Date}
}

\usepackage{hyperref}
\usepackage{listings}

\definecolor{dkgreen}{rgb}{0,0.6,0}
\definecolor{gray}{rgb}{0.5,0.5,0.5}
\definecolor{mauve}{rgb}{0.58,0,0.82}

\renewcommand{\lstlistingname}{Code Example} % a listing caption title.
%\renewcommand{\lstlistlistingname}{List of \lstlistingname s} % list of lists -> list of Thread Program
\lstset{
    frame=single,
    language=C,
    columns=flexible,
    numbers=left,
    numbersep=5pt,
    numberstyle=\tiny\color{gray},
    keywordstyle=\color{blue},
    commentstyle=\color{dkgreen},
    stringstyle=\color{mauve},
    breaklines=true,
    breakatwhitespace=true,
    tabsize=4,
    captionpos=b
}

\def\name{Put your name here}

%% The following metadata will show up in the PDF properties
\hypersetup{
  colorlinks = false,
  urlcolor = black,
  pdfauthor = {\name},
  pdfkeywords = {Capstone Tech Review},
  pdftitle = {Capstone Tech Review},
  pdfsubject = {Tech Review},
  pdfpagemode = UseNone
}

\setlength{\parindent}{0.0 in}
\setlength{\parskip}{0.2 in}

\begin{document}

\begin{titlepage}
\title{
Group 44 - ARC \\
Tech Review\\
\LARGE
Senior Capstone Project\\
Oregon State University\\
Fall 2016
}

\author{Tao Chen, Cierra Shawe, Daniel Stoyer}
\maketitle

\begin{abstract}

\end{abstract}

\thispagestyle{empty} % gets rid of the "0" page number.

\end{titlepage}
\newpage

\tableofcontents

\newpage

\section{Vision System Options - Cierra} % Done
For autonomous operation, vision systems are critical. 
The three main options include stereoscopic cameras, Infrared (IR) based systems such as Microsoft's Kinect, and Light Detection And Ranging (LiDAR) vision systems. 
With the exception of some forms of LiDar, all of these methods require what is called a disparity map, which creates a 3D image of the surface, that can be used for telling which objects in an image are closest or farthest away. 

\subsection{Stereo Vision}
Stereo-vision uses two different cameras to create disparity maps in order to create a sense of depth. 
This is similar to how our eyes work. 
The biggest benefit to a stereoscopic camera system is the ability to detect objects outdoors, as the cameras are able to function with vast amounts of ultraviolet (UV) light. 
IR LEDs can also be used in order to illuminate an area at night, also allowing for night-time navigation. 
Another pro to stereoscopic vision, is the cost is relatively low, as cameras can be obtained for under \$10 a piece.  
One of the challenges of stereo vision is the computational power required. 
Another is clarity of the disparity map without post processing of images, which makes real-time operation more difficult. 
\cite{acs}
OpenCV \cite{opencv} contains many examples of how to configure and process stereoscopic images, and is one of the largest vision resources. It can be used to synchronize and create the the disparity map, which can then be used by another part of our system for decision making. 

\subsection{IR Camera's such as Kinect and RealSense}
Using an infrared point map, these cameras are able to tell the disparity between the points, which helps in creating disparity maps. The most popular example of an IR camera system, is the Microsoft Kinect. 
A big advantage to IR camera's, is the ability to function in low and non-natural lighting conditions, due to using the infrared spectrum, rather than only using the visible spectrum. 
The biggest problem with IR cameras, the functionality is greatly reduced outdoors, due to massive amount of infrared waves from the sun. 
IR cameras don't meet our requirement of being able to use the vision system reliably outdoors.

\subsection{Lidar}
LiDAR works by using laser pulses to detect range. 
By detecting different pulse signatures, it is able to take very precise distance measurements. 
LiDAR is a great way to form point clouds, however, the cost makes the product unreasonable for most people. 
A basic SICK LiDAR unite costs around \$2,000 USD for a unit with 10m accuracy. cite{sick} https://www.sick.com/us/en/product-portfolio/detection-and-ranging-solutions/2d-laser-scanners/tim3xx/c/g205751\\
A 2D RPLiDAR module is a lower cost alternative, at \$449, however, the manufacture says that it will not preform well in direct sunlight, due to using IR lasers. http://www.robotshop.com/en/rplidar-a2-360-laser-scanner.html
The RPLiDAR also only has accurate measurements up to 6 meters.\\
A third option for LiDAR, has yet to come to market. 
The Sweep LiDAR unit by Scanse will be released in January of 2017. 
The Sweep unit claims to be a LiDAR unit available for all.
Costing \$349, the unit has the ability to scan 360 degrees, create points up to 40m away, and is able to function in "noisy" environments, such as outdoors. \\
The Sweep's distance capabilities and ability to be used almost any lighting condition, including outdoors, would make this a great candidate for our project.
The caveat to the Sweep, is it will not be released until January. \par
The LiDAR systems that would be available to our group, would not meet the requirement of being able to function consistently in an outdoor environment, or do not fall under the category of commodity hardware. 

\subsection{Our choice}
Due to the need to be able to navigate in outdoor environments, our team will start out attempting to use stereoscopic imaging as our primary vision system. We will do this using the OpenCV library to analyze the images, and create the disparity map that can be used for other purposes. If we have the computational power to post-process disparity maps in real time, we will attempt to do so. \\
If time allows, given we are able to obtain a unit, our team could investigate the use of the Sweep LiDAR system for navigation in place of stereoscopic imaging. 

\section{Sensors - Cierra}
\subsection{GPS - External}
\subsection{GPS - PXFmini}
\subsection{Internal Measurement Unit (IMU) - External}
\subsection{IMU - PXFmini}


\section{System Control and Synchronization - Cierra}
IDK WHAT TO PUT IN THIS SYSTEM
\subsection{Intel NUC}

\subsection{Intel UPboard}

\subsection{Raspberry Pi 3}

\newpage

\section{Image Analysis Software - Dan}
Image analysis, for the ARC project, is the processing of visual data received
from cameras into deterministic information, such as path-finding, or spacial
awareness. This is the primary means for our autonomous vehicle to assess its
surroundings and find its way to a given way-point while avoiding obstacles. We
require software that is freely available for use (via fairly liberal open
source licensing), known to be correct (works well) with little modification
needed, and has relatively easy to use API libraries.

\subsection{Option 1}
more to follow...

\subsection{Option 2}
more to follow...

\subsection{Option 3}
more to follow...

\subsection{Image analysis choice}
ArduPilot is our choice for image analysis software. The documentation for the
image analysis software researched was rather vague across the board when it
comes to information on path-finding and image analysis capabilities. So, while
we are going with ArduPilot to start with, we will not really know its effective
capability at obstacle avoidance and path-finding until we have tested it.

\subsection{References}

\newpage

\section{Telemetry Radio Communication - Dan}
In this section we will examine three different telemetry radios, comparing and
contrasting them and making a choice on which radio we will use for ARC.
Telemetry is simply the transmission of measurement data (velocity, angle,
rotation, etc.) by radio to some other place. [1] This data allows the user to know the current state of the vehicle. This is especially
important for autonomous operation, as the vehicle may not be operating within
line of sight. Telemetry transmission is
well-established, so we will not be comparing vastly different transmission
technologies, such as long range (MHz radio frequencies) versus short-range
(blue-tooth) where the advantages of ranges of 2-15+ kilometers obviously
outweigh ranges of 20-100 meters.

The main criteria for consideration are:
\begin{itemize}

	\item Cost
		\subitem One of our main goals with ARC is to keep the costs low.
	\item Power consumption
		\subitem We have limited power available, therefore we need power
		consumption to be low.

	\item Ease of use
		\subitem The radio needs to be easily integrated into the autopilot
		system. This means it needs to have a developed API with little no
		modification required.

	\item Form factor
		\subitem The size and weight needs to be small and light. If it is too
		bulky, we might not have space on the vehicle. If it is too heavy, more
		power will be required to operate the drive system and will drain the
		battery faster.

\end{itemize}


\subsection{3DR 915 MHz Transceiver}

The 3DR 915 MHz telemetry radio has a cost of \$39.99 USD for two radios. It is
powered by the autopilot telemetry port (+5v) which means it has low power
consumption. This radio transceiver uses open source firmware, has a robust API,
and is fully compatible with PX4 Pro, DroneKit, and ArduPilot, using the MAVLink
protocol. These features will allow us to implement telemetry transmission with
little to no modification of the API, should we use one of those autopilot
systems.  The form factor has dimensions of 25.5 x 53 x 11 mm (including case,
but not antenna) at 11.5 grams (without antenna). [2]

The range of this transceiver is from 300 meters to several kilometers,
depending on the antenna arrangement.

Pros: inexpensive, small form factor, low power consumption.

Cons: range out of the box could be as low as 300 meters.


\subsection{RFD900 Radio Modem}

The RFD900 Radio Modem has a cost of \$259.99 USD for two radios. [3] It requires
separate +5v power for operation which means that it has high power
consumption. This radio has open source firmware, a robust API, and is fully
compatible with PX4 Pro, DroneKit, and ArduPilot, using the MAVLink protocol,
which will allow us to implement telemetry transmission with little to no
modification of the API, should we use one of those autopilot systems. [4]

The form factor has dimensions of 70 x 40 x 23mm (including case, but not
antenna) at 14.5 grams (without
antenna).
The range of this transceiver is 25+ kilometers.

Pros: ultra long range.

Cons: expensive, large size.


\subsection{Openpilot OPLink Mini Ground and Air Station 433 MHz}

The OPLink Mini Ground Station has a cost of \$26.59 USD for two radios. [5] It
requires input voltage of +5v and can be powered off the autopilot telemetry
port which means that it has low power consumption. This radio has open source
firmware but is only compatible with the OpenPilot RC control system. 
The form factor has dimensions of 38 x 23 x mm (including case, but not
antenna) at 4 grams (without antenna). [6]
The range of this radio is not known, but based on the power requirements and
frequency it likely has less range than the 3DR 915 MHz radio.

Pros: smallest size and weight (only 4 grams), lowest cost (\$26.59 USD)

Cons: Only works with the LibrePilot control system.


\subsection{Telemetry radio choice}
The 3DR 915 MHz Transceiver is our selection for the telemetry radio.
While the OPLink Mini Ground Station was significantly smaller, lighter, and
cheaper than the other two, its implementation being tied solely to LibrePilot
was a deal breaker (more information on LibrePilot can be found in the User
Interface evaluation).
The RF900 Radio Modem would have been a good choice, it has fantastic range
and all the API options we were looking for. But it had a significantly larger
form factor, required a separate power supply, and was quite expensive at
\$259.99. Put together, these facts eliminated the RF900 as a viable option.
The 3DR 915 MHz Transceiver is a good balance of cost, performance, and size. The
cost of \$39.99 for two radios, the ability to power the autopilot off the
telemetry port, and the portability of its APIs and their ease of use, puts the
3DR 915 MHz at the top of our list and the clear choice for the telemetry
radio going forward.

\subsection{References}

[1] Merriam-Webster.com, 'telemetry', 2016. [Online]. Available: http://www.merriam-webster.com/dictionary/telemetry. [Accessed: 15- Nov- 2016].\par

[2] 3DR, '915 MHz (American) Telemetry Radio Set', 2016. [Online]. Available: https://store.3dr.com/products/915-mhz-telemetry-radio. [Accessed: 15- Nov- 2016].\par

[3] jDrones.com, 'jD-RF900Plus Longrange', 2016. [Online]. Available: http://store.jdrones.com/jD\_RD900Plus\_Telemetry\_Bundle\_p/rf900set02.htm. [Accessed: 15- Nov- 2016].\par
[4] ArduPilot Dev Team, 'RFD900 Radio Modem', 2016. [Online]. Available: http://ardupilot.org/copter/docs/common-rfd900.html. [Accessed: 15- Nov- 2016].\par

[5] Banggood.com, 'Openpilot OPLINK Mini Radio Telemetry', 2016. [Online].
Available: http://www.banggood.com/Openpilot-OPLINK-Mini-Radio-Telemetry-AIR-And-Ground-For-MINI-CC3D-Revolution-p-1018904.html [Accessed: 15- Nov-2016].\par

[6] HobbyKing.com, 'Openpilot OPLink Mini Ground Station 433 MHz', 2016.  [Online]. Available: https://hobbyking.com/en\_us/openpilot-oplink-mini-ground-station-433-mhz.html [Accessed: 15- Nov-2016].\par

\newpage

\section{User Interface - Dan}
In this section we will examine three user interfaces, comparing and contrasting
them and making a decision on which one we will use with ARC.
A user interface (UI) is required to allow the user to command the vehicle. The
UI must be open source and have easy-to-implement API libraries. We are looking
for a UI package that will work with both the control station (the user
computer) and the companion computer (the computer on board the vehicle). It is
preferable that the UI be a combination of graphical UI (GUI) and command line
UI (CLI). Note that though our project is a land vehicle (rover) the following
software is primarily used for UAV flight control and is referenced in such a
way. If possible, we would like to use software that can be configured to
control a rover, or easily modified to do so.

\subsection{QGroundControl}

QGroundControl (QGC) is a full flight control and mission planning GUI software
package that is compatible with any MAVLink enabled drone. [1] It is open source and
is configured for use with ArduPilot and PX4 Pro. QGC runs on Windows, OS X,
Linux, and iOS and Android tablets. QGC has video streaming with instrument
overlays, allows mission planning including map point selection, rally points,
and even a virtual fence to keep the drone from going beyond a specified area.
QGC is a mature software package that has excellent libraries and support with
very good documentation. [2]
Additionally, QGC works with ArduPilot which is known to work with the PXFMini ,
the autopilot we intend to use and can be configured for rovers. [3] QGC appears
be GUI only with no CLI functionality.

Pros: easy to use, great documentation, compatible with MAVLink, tested on the
PXFMini. Can be used on all major pc and mobile platforms. Supports rovers.

Cons: does not appear to have CLI support.

\subsection{Tower/DroneKit-Android}(http://android.dronekit.io/) (https://play.google.com/store/apps/details?id=org.droidplanner.android\&hl=en)
Tower is a Android mobile app that works with most drones that use the MAVLink
protocol. Tower allows basic map point selection and allows drawing a flight
path on the tablet. [4] It is based on the open source DroneKit-Android framework.
DroneKit-Android has good documentation providing code snippets with working
example code. [5] Because of the modularity of Android development and access to
Tower source code, adding feature and interfaces to the existing app should be
relatively easy. Tower is not configured for rovers, so we would have to write
the functionality into it. DroneKit-Android and Tower are only available on Android. 

Pros: is easy to use, has basic map point selection, adding features should be
relatively straight-forward.

Cons: is only available on Android, configuring for rovers requires writing code
for support.

\subsection{LibrePilot}
(https://www.librepilot.org/site/index.html)  

LibrePilot (LP) is a full flight control and mission planning software
package. It is open source and operates via GUI and allows map point selection.
LP is compatible with OpenPilot control system exclusively. It does not
work with any other hardware but OpenPilot hardware and does not have rover
support. It has some helpful documentation, such as Windows build instructions,
but the information is very basic. The source code does seem to have decent
commenting which could help since we would need to heavily modify the code base
for rover support. LP runs on Linux, Mac, Windows, and Android. [6]

Pros: Has a nice GUI for map point selection, mission planning, and vehicle
control. If used in the OpenPilot ecosystem, it should communicate well. Runs
on most major pc platforms.

Cons: Is locked in to the OpenPilot ecosystem. Does not have rover support out
of the box which will require extensive coding. Does not run in iOS.

\subsection{User interface choice}
QGroundControl is our user interface choice.\par
LibrePilot has similar features to QGC but being locked in to the OpenPilot
ecosystem is a deal breaker. We need to be able to use the PXFMini and
LibrePilot cannot do that. LP is also not configured for rovers, which would
require extensive coding.\par
Tower is the most modest of the user interface options. It is only available on
Android, does not have rover support and has limited options for navigation and
vehicle control. For these reasons we reject Tower as a viable option.\par
QGroundControl runs on all major pc and mobile platforms, is configured to run
rovers, and uses the MAVLink protocol. It has a nice GUI and allows map point
selection and advanced mission planning. It is known to work with the PXFMini
flight controller, a component we want to use as part of the ARC build. These
features give us a platform that is meets our needs and is flexible, should our
needs change. Therefore, QGroundControl is the clear user interface choice for
the ARC project.

\subsection{References}

[1] QGroundControl.com, Unknown. [Online]. Available: http://qgroundcontrol.com/. [Accessed: 15- Nov- 2016].\par

[2] QGroundControl.com, Unknown. [Online]. Available: https://donlakeflyer.gitbooks.io/qgroundcontrol-user-guide/content/. [Accessed: 15- Nov- 2016].\par

[3] ArduPilot Dev Team, 'PXFmini Wiring Quick Start', 2016. [Online]. Available: http://ardupilot.org/rover/docs/common-pxfmini-wiring-quick-start.html.  [Accessed: 15- Nov- 2016].\par

[4] Fredia Huya-Kouadio, 'Tower', 2016. [Online]. Available: https://play.google.com/store/apps/details?id=org.droidplanner.android.  [Accessed: 15- Nov- 2016].\par

[5] 3D Robotics Inc., 'DroneKit', 2015. [Online]. Available: http://dronekit.io/. [Accessed: 15- Nov- 2016].\par

[6] LibrePilot, 'Open-Collaborative-Free', 2016. [Online]. Available: https://www.librepilot.org/site/index.html. [Accessed: 15- Nov- 2016].\par

\newpage

\section{Control System - Tao}
A good control system can improve the accuracy of the estimated current location 
of the vehicle. If we choose unreliable control scheme and motion model, it will 
produce a lot of overhead for our main computer and the results won’t be as precise. 
Power consumption may increase due to the excessive amount of calculation done by 
the computer. In conclusion, motor control, servo control, and motion model are 
methods to minimax the computation the main computer has to undergo.

\subsection{Motor Control}
There are two types of motor control schemes that are practical for our project. 
I would not say that they are technologies. They are just two ways to decide 
how the vehicle should move forward and backward:
\begin{enumerate}
\item Time Critical: This motor control scheme tells the motor to spin 
forward/backward for a certain amount of time at a certain speed.
\item Distance Critical: This motor control scheme tells the vehicle to go 
forward/backward for a certain distance.
\end{enumerate}
When an iteration is finished under time critical, the system will decide whether 
to keep the speed for another time period or slow down or accelerate. In order 
to reduce jerkiness, the rate of acceleration and deceleration will be low.\par

When an iteration is finished under distance critical, the system will decide what 
to in the next cycle. It can either switch to time critical or move for another 
distance.\par

In the final implementation, we may switch between the two methods based on 
real-time conditions. When the vehicle is operating at high speed in an open 
environment, such as a parking lot, with GPS cooperating, time critical better 
suits my purpose. When operating indoor without a pre-defined map, in other words, 
the vehicle is exploring the environment, distance critical works better. When the 
system detects an approaching obstacle, whether it is operating indoor or outdoor, 
the system should always switch to distance critical.\par

\subsection{Servo Control}
Servos control the steering of the vehicle. Servo control is similar to motor control. 
There are also two scheme under which the system operates the servo:
\begin{enumerate}
\item Time critical: The system tells the servo to keep a certain angel for a certain 
amount of time.
\item Angel critical: The system tells the vehicle to steer left/right for a certain 
angel. 
\end{enumerate}
Time critical makes drifting possible, which it is one of the goals of this project. 
When the car is operating at high speed, time critical will be easier to harness because 
oversteering will happen, and angel critical will cause unexpected maneuver when 
oversteering happens.\par

Angel critical may do a better job in obstacle avoidance. The angle of turning is defined 
as the angel between the driving direction of vehicle before and after the turn.\par

Again, in the final implementation, both schemes will be implemented and the system will 
switch between them based on real-time situation.\par

\subsection{Probabilistic Analysis for Motion}
\begin{enumerate}
\item Bayes Filter
\item (Extended) Kalman Filter
\item Particle Filter
\end{enumerate}

\subsection{Motion Model}
In this context, the motion model states the behavior of the vehicle under different 
combinations of speed and steering. A concrete and precise motion model is important to 
the control system. The control system operates the vehicle based on this pre-defined 
motion model. For example, if the system needs to turn the vehicle 90 degrees right, it 
will have to output a sequence of actions by applying the motion model to the status of 
the vehicle, such as speed and center of gravity.\par

Different vehicle will have different configurations of motion model. When our vehicle 
is fully loaded with hardware, it will also have a different configuration of motion 
model than when it is unloaded. Thus, we will have to conduct multiple experiments to 
create the motion model of our vehicle. Alternatively, ROS provides a very nice simulation 
environment, where we can have a script that describes the configuration of the vehicle 
and have the simulator run the vehicle.\par

Potentially, as the development progresses, we may find the motion model to be obsolete, 
because I have a feeling that we can get around without a pre-defined motion model. But 
for now, a large part of me still believes it is necessary.\par

\section{Path Planning - Tao}
\subsection{Global Path Planning}
Global in the perspective of our vehicle will be an area about the same size of a 
basketball court. We will pre-define maps with multiple way points for the system to 
navigate itself through.\par
\begin{itemize}
\item Breath-first Search:\par
This algorithm promises to output the optimal path for start to destination in terms 
of nodes visited. In reality, the cost of going from one node to another should be 
considered when planning, so this algorithm will not be considered. 
\item Depth-first Search:\par
Similar to breath-first search, this algorithm guarantees to find the optimal path 
for start to destination in terms of nodes visited. In reality, the cost of going 
from one node to another should be considered when planning, so this algorithm will 
not be considered. 
\item Dijkstra’s algorithm:\par
This is algorithm and breath-first search are very much alike. This algorithm takes 
into account the costs between nodes and promises to find the shortest path.
\item A* Heuristic search:\par
This algorithm uses an extra variable, the heuristic, to improve the performance 
of the Dijkstra’s algorithm. 
\end{itemize}
In conclusion, we will be using the A* Heuristic search algorithm as our global 
path planning algorithm. It produces reliable results and is the fastest algorithm 
among all the above-mentioned algorithms. Our computer should handle the computation 
no problem.\par  

\subsection{Local Path Planing}
local in the perspective of our vehicle will be an area about the same size of a 
classroom. The vehicle may have to build the map itself by exploring the space or 
be provided with a pre-defined map.\par
\begin{itemize}
\item Rapidly-Exploring Random Trees (RRTs):\par
This algorithm guarantees to find a path from start to goal as the number of 
sample points goes to infinity.
\item RRT*:\par
This algorithm is an extension of RRTs. It promises to find the optimal path from 
start to goal as the number of sample points goes to infinity. 
\end{itemize}
These two algorithms are both feasible given the map is small, because they require 
a large amount of data to be stored in memory. This is why we did not consider the 
two to be our global path planning algorithms.\par

Depending on the computer memory and time constraints, we will use any one of them 
that suits our purpose. If the system is asked to output a valid path as fast as 
possible, RRTs will be used. If we were to ask the system to output the shortest path 
it can generate given all the computational resources, RRT* will be used.\par

\section{Other Algorithms - Tao}
\subsection{Obstacle Avoidance Algorithm}
In the global and local path planning algorithms sections, the underlying assumption 
is that the world is static (the map does not change). However, our team wants the 
vehicle to respond properly when unexpected objects are blocking the planned path.\par 

There are two ways to accomplish this. We can:\par
\begin{enumerate}
\item Diverge from the original path as little as possible and converge back to 
the pre-planned path as soon as possible.
\item Reconstruct an entirely new path.
\end{enumerate}

While re-planning may be more intelligent overall, it is infeasible in a global 
setup due to the large amount of computation the system has to redo. Rarely the 
construction of a path will take less than noticeable period time. Our team’s 
vision of the system is to enable the vehicle to drive itself as if a rational 
human is driving it. New plan construction that causes suspension of movements is 
unacceptable. Thus, we will go for the first option and will not consider 
reconstructing new paths.\par

\subsection{Parallel Parking Algorithm}
By going through a research project carried out by students at University of 
Minnesota, parallel parking requires the system to perform following tasks:\par

\begin{enumerate}
\item Parking Spot Detection
\item Parking
\item Exiting the spot
%%\item (http://www-users.cs.umn.edu/~joel/_files/Joel_Hesch_EE4951.pdf)
\end{enumerate}

For simplicity, we don’t need to perform parking spot detection if the system 
is given a map. The algorithm will output a trajectory to the control system on 
every iteration. The system will be using multiple sensors, such as IR/sonar 
sensors, to scan surroundings, adjusting the trajectory accordingly. The control 
system must make sure the vehicle follows the trajectory precisely.\par


\end{document}
