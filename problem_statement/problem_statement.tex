\documentclass[draftclsnofoot,onecolumn,10pt]{IEEEtran}
\usepackage[utf8]{inputenc}
\usepackage{color}
\usepackage{url}

\usepackage{enumitem}

\usepackage[letterpaper, margin=.75in]{geometry}

\newcommand{\toc}{\tableofcontents}
\newcommand*{\SignatureAndDate}[1]{
    \par\noindent\makebox[2.5in]{\hrulefill} \hfill\makebox[2.0in]{\hrulefill}
    \newline\noindent\makebox[2.5in][l]{#1}  \hfill\makebox[2.0in][l]{Date}
}

\usepackage{hyperref}
\usepackage{listings}

\definecolor{dkgreen}{rgb}{0,0.6,0}
\definecolor{gray}{rgb}{0.5,0.5,0.5}
\definecolor{mauve}{rgb}{0.58,0,0.82}

\renewcommand{\lstlistingname}{Code Example} % a listing caption title.
%\renewcommand{\lstlistlistingname}{List of \lstlistingname s} % list of lists -> list of Thread Program
\lstset{
    frame=single,
    language=C,
    columns=flexible,
    numbers=left,
    numbersep=5pt,
    numberstyle=\tiny\color{gray},
    keywordstyle=\color{blue},
    commentstyle=\color{dkgreen},
    stringstyle=\color{mauve},
    breaklines=true,
    breakatwhitespace=true,
    tabsize=4,
    captionpos=b
}

\def\name{Put your name here}

%% The following metadata will show up in the PDF properties
\hypersetup{
  colorlinks = false,
  urlcolor = black,
  pdfauthor = {\name},
  pdfkeywords = {Capstone problem statement autonomous},
  pdftitle = {Capstone Problem Statement},
  pdfsubject = {Problem Statement},
  pdfpagemode = UseNone
}

\setlength{\parindent}{0.0 in}
\setlength{\parskip}{0.2 in}

\begin{document}

\begin{titlepage}
\title{
ARC Problem Statement\\
\LARGE
Senior Capstone Project\\
Oregon State University\\
Fall 2016
}

\author{Tao Chen, Cierra Shawe, Daniel Stoyer}
\maketitle

\begin{abstract}
Currently, autonomous vehicles are built with many custom parts and assemblies and are very expensive. 
The cost of entry to autonomous vehicles is often prohibitive for the average consumer. 
The purpose of the Autonomous RC (ARC) project is to determine if it is possible to build an autonomous RC vehicle using commodity hardware.  
This will be accomplished using commodity hardware, including cameras, computers, and an RC vehicle used as the base, with minimal fabrication. \\

\vspace{2cm}

\SignatureAndDate{D. Kevin McGrath} \newline\newline
\SignatureAndDate{Tao Chen}\newline\newline
\SignatureAndDate{Cierra Shawe}\newline\newline
\SignatureAndDate{Daniel Stoyer}\newline\newline
\end{abstract}

\thispagestyle{empty} % gets rid of the "0" page number.

\end{titlepage}
%\newpage

%\tableofcontents

\newpage

\section{Motivation}
Millions, if not billions of dollars have been allocated for research of autonomous vehicles. 
A lot of the hardware is very specialized, and often very expensive. 
Our challenge is to see if it is possible  at a small-scale to replicate a subset of the expensive system's abilities.
Our goal is to to use off-the-shelf retail/commodity hardware (i.e. something you can purchase from Best Buy, Radio Shack, Amazon, etc.) that costs hundreds of dollars instead of hundreds of thousands. 
Being able to implement even a small subset of the features at a small-scale, such as a vision system, will potentially mean a feature can also be scaled for larger implementations. 
A small-scale autonomous vehicle (AV), built with commodity hardware, will provide an inexpensive way to test autonomous features and the viability of new research in the autonomous vehicle field, even for a hobbyist user.

\section{Problem Description}
Due to the expense of autonomous vehicles, there are few options available that can be easily retrofitted by the average user. 
Our goal is to find out whether or not it is possible to take commodity hardware, retrofit it onto a remote controlled (RC) car, and have it be able to navigate obstacles at speed. 
We also want to create a list of hardware components and a software solution that can be retrofitted onto at least one model of RC car, in order to enable autonomous navigation and operation. 
Ideally,  our solution can be used on any RC car, with minimal modification. 
As this is a research project, we do not know if it is actually possible to implement these features using commodity hardware.\par
Our implementation will be a small-scale version of the Defense Advanced Research Projects Agency (DARPA) Grand Challenge, where autonomous vehicles navigate themselves 100 miles through the Mojave Desert, in under 10 hours. 
Our goal is to implement an RC car able to navigate to a given point as fast as possible, which is a small-scale version of the challenge. 
The RC car should be able to navigate within a set amount of time through a room or defined map based area, without prior knowledge of the environment.
The car should also be able to parallel park. 
Stretch goals for the project are high-speed obstacle avoidance and the ability to drift (where the wheels break traction on a surface) on a dirt track.

\section{Proposed Solution}
Our solution will prove if it is possible to implement autonomous point-to-point navigation of an RC vehicle in low-speed and high-speed scenarios using off-the-shelf retail/commodity hardware.  
The software will be an implementation of readily available open-source distributions, such as Georgia Tech's GT AutoRally. 
For hardware, we will use a retail RC vehicle as the base with additional hardware for computation, such as the Intel NUC micro-PC, and control, such as the Raspberry Pi with a PixHawk Fire Cape. 
The major components including camera systems, GPS, accelerometers, etc. will be kept as close to retail as possible. 
Certain hardware, such as mounts, may need to be fabricated to attach components, such as the mini-PC, to the RC vehicle.\par
Our presentation at Engineering Expo will be a booth that will either show successful autonomy of an RC vehicle or show why it is not yet possible to produce the autonomous requirements specified in the problem definition. 
The presentation will include a large poster with the team/project name and smaller displays depicting elements of the project process. 
There will be a video comprised of the final capstone presentation and also footage of the project taken throughout the term and will have basic editing and formatting for appearance. 
We will have our final capstone report available for those interested and will display the RC vehicle used in the project. 
The project team members will accompany the Engineering Expo presentation and will interact with those interested.

\section{Performance Metrics}
This project will be successful if the RC vehicle is able to autonomously navigate point-to-point using low-speed obstacle avoidance and high-speed maneuvering when appropriate. 
Examples of low-speed scenarios are ninety-degree corners, tight s-turns, or multiple offset obstacles in the path. 
High-speed scenarios will include straight-line acceleration and drifting through large-radius corners. 
An unsuccessful result, if accompanied by enough data and documentation, will show it is not yet possible to implement autonomy at the RC vehicle scale using only off-the-shelf/commodity hardware.\par
Basic Criteria:
\begin{itemize}
	\item Self-navigate from point to point.
	\item Low-speed obstacle avoidance
	\item Parallel parking
\end{itemize}
 Advanced Criteria and Stretch Goals:
 \begin{itemize}
	\item Obstacle avoidance at high speed.
	\item Drifting (Dust track) 
	\item Track current position with respect to the location of the base station in
		order to not have the car drive away.
\end{itemize}
At the end of Spring 2017 (before the Engineering Expo), all the basic objectives above should be accomplished on the RC vehicle provided, with minimal specialized hardware and fabrication. 
Advanced Criteria and Stretch goals will be used to provide additional criteria if the basic criteria are finished early.
If a final product is not delivered on time (before the Engineering Expo), we shall provide strong reasonings on why an objective was not feasible with the time and given resources.

The outcome of this project will be measured on the completeness of the final product or the soundness of the reasonings if any of the objectives are not achieved.
\end{document}
