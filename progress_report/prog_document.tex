\documentclass[compsoc,draftclsnofoot,onecolumn,10pt]{IEEEtran}
\usepackage[utf8]{inputenc}
\usepackage{color}
\usepackage{url}

\usepackage{enumitem}

\usepackage[letterpaper, margin=.75in]{geometry}
\usepackage{hyperref}
\usepackage{listings}

\usepackage[dvipsnames]{xcolor}
\usepackage{pgfgantt}
\newcommand*{\SignatureAndDate}[1]{
    \par\noindent\makebox[2.5in]{\hrulefill} \hfill\makebox[2.0in]{\hrulefill}
    \newline\noindent\makebox[2.5in][l]{#1}  \hfill\makebox[2.0in][l]{Date}
}
\usepackage{etoolbox}
\patchcmd{\thebibliography}{\section*}{\subsection}{}{}
\patchcmd{\thebibliography}{\addcontentsline{toc}{section}{\refname}}{}{}{}

\usepackage{graphicx}

\definecolor{dkgreen}{rgb}{0,0.6,0}
\definecolor{gray}{rgb}{0.5,0.5,0.5}
\definecolor{mauve}{rgb}{0.58,0,0.82}

\renewcommand{\lstlistingname}{Code Example} % a listing caption title.

\lstset{
	frame=single,
	language=C,
	columns=flexible,
	numbers=left,
	numbersep=5pt,
	numberstyle=\tiny\color{gray},
	keywordstyle=\color{blue},
	commentstyle=\color{dkgreen},
	stringstyle=\color{mauve},
	breaklines=true,
	breakatwhitespace=true,
	tabsize=4,
	captionpos=b
}

\def\name{Cierra Shawe, Daniel Stoyer, Tao Chen}

%% The following metadata will show up in the PDF properties
\hypersetup{
	colorlinks = false,
	urlcolor = black,
	pdfauthor = {\name},
	pdfkeywords = {SRS requirements, Fall 2016},
	pdftitle = {SRS requirements},
	pdfsubject = {Requirement document for ARCt},
	pdfpagemode = UseNone
}

\def\myversion{1.0 }
\date{}
%
%\usepackage{titlesec}
%
%\usepackage{hyperref}

\parindent = 0.0 in
\parskip = 0.1 in

\setcounter{secnumdepth}{5}

\begin{document}

\begin{titlepage}
\title{
Progress Report\\
\LARGE
ARC - Autonomous RC\\
Senior Capstone Project\\
Oregon State University\\
Fall 2016
}
\author{Tao Chen, Cierra Shawe, Daniel Stoyer}
\maketitle
\begin{center}
	Version 1.0\\
	\today
\end{center}

\thispagestyle{empty} % gets rid of the "0" page number.
	
\end{titlepage}

\tableofcontents

\newpage

\section{Project purpose and goals} 
% Breifly recaps project purpose and goals

\section{Current Status}
% Describes where we currently are on the project.

\section{Week-by-week summary of activities}

\subsection{Weeks 1 - 3}
Weeks one through three were general introduction and orientation weeks. It was
not until week 4 that projects started in earnest.

\subsection{Week 4}
	\begin{itemize}
        \item \textit{Activities}:\\
        Fill me in!
        \item \textit{Problems}:\\
        Fill me in!
        \item \textit{Solutions}:\\
        Fill me in!
	\end{itemize}
   
\subsection{Week 5}
	\begin{itemize}
        \item \textit{Activities}:\\
        Fill me in!
        \item \textit{Problems}:\\
        Fill me in!
        \item \textit{Solutions}:\\
        Fill me in!
	\end{itemize}
   
\subsection{Week 6}
	\begin{itemize}
        \item \textit{Activities}:\\
        Fill me in!
        \item \textit{Problems}:\\
        Fill me in!
        \item \textit{Solutions}:\\
        Fill me in!
	\end{itemize}
   
\subsection{Week 7}
	\begin{itemize}
        \item \textit{Activities}:\\
        Fill me in!
        \item \textit{Problems}:\\
        Fill me in!
        \item \textit{Solutions}:\\
        Fill me in!
	\end{itemize}
   
\subsection{Week 8}
	\begin{itemize}
        \item \textit{Activities}:\\
        Fill me in!
        \item \textit{Problems}:\\
        Fill me in!
        \item \textit{Solutions}:\\
        Fill me in!
	\end{itemize}
   
   
\subsection{Week 9}
	\begin{itemize}
        \item \textit{Activities}:\\
        Fill me in!
        \item \textit{Problems}:\\
        Fill me in!
        \item \textit{Solutions}:\\
        Fill me in!
	\end{itemize}
   
   
\subsection{Week 10}
	\begin{itemize}
        \item \textit{Activities}:\\
        Fill me in!
        \item \textit{Problems}:\\
        Fill me in!
        \item \textit{Solutions}:\\
        Fill me in!
	\end{itemize}
   
   
\section{Retrospective}
	\begin{center}
		\begin{tabular}{|p{0.3\linewidth}|p{0.3\linewidth}|p{0.3\linewidth}|}
			\hline
			\textbf{Positives:} & \textbf{Deltas:} & \textbf{Actions:}\\
			
            Anything good that happened. & Changes that need to be implemented. & Specific actions to counter deltas.\\
			\hline

			dummy positives & dummy negatives & dummy actions\\
			\hline
			more dummy positives & moredummy negatives & more dummy actions\\
			\hline
			
		\end{tabular}
	\end{center}

\end{document}
